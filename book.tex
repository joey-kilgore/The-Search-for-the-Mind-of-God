\documentclass[12pt,a4paper]{book}
\usepackage{marginnote}
\usepackage{comment}
\usepackage{graphicx}
\usepackage{hyperref}

\hypersetup{
    colorlinks,
    citecolor=black,
    filecolor=black,
    linkcolor=black,
    urlcolor=black
}

\title{The Search for the Mind of God \\ \large Pursuing a PhD and Christ \\ v0.2}
\author{Joseph Kilgore}
\begin{document}
\frontmatter
\maketitle

\tableofcontents
\mainmatter

\chapter{The Narrative}
One thing that has been made readily apparent in the world of immediate connectivity is that people are more than willing to explain why they are right and others are wrong, and for the purposes of this book, it can sometimes revolve around beliefs of science and faith. Most people have their stock points ready and waiting to take down any arguments brought against them. A fundamental problem arises for humanity, do we actually listen to the other side?

The stereotypical argument of science vs. faith is something like the following:

\begin{flushleft}
Scientist: Faith is just an excuse for not understanding the explainable universe.

\medskip 
Christian: No, Christianity is understanding the miracles that God has done in the world and is continuing to do today.

\medskip 
Scientist: Miracles are a physical impossibility. You cannot simply break the laws of the universe on a whim. We have proven that we don't need a God to explain the creation of the universe, and we certainly don't need a God to explain the creation of humanity.

\medskip 
Christian: You cannot prove that God didn't create the universe because He exists outside our universe. There is no way to measure Him.

\medskip 
Scientist: What do you know? You aren't a scientist! You don't know how to explain the workings of the universe like I do!

\medskip 
Christian: What do you know? You aren't a Christian! You don't know how to explain the workings of the universe like I do!
\end{flushleft}

Neither side has done a good job at really working to understand the opposition. Instead of the intent to convert we need an intent to reach a mutual understanding. Books like \textit{The Grand Design} by Stephen Hawking give a perfect representation of scientists trying to convert the weak-minded masses from their dogmatic anti-scientific ideologies \cite{hawking_mlodinow_2010}. And while scientists can certainly do this, I think Christians can do this far more often. The hard part with Christians doing this is that they can think that this constant push of Christianity is their way of proclaiming their faith, and while this can be true, we need to think twice about whether we are showing people the love of Christ or trying to push Christianity down their throats. We as Christians need to remember that Christ didn't force belief in Him onto all of those around Him. They were free to believe or to disbelieve. It's a fine line to walk of sharing without pushing, but I think we need to be more mindful of it.

So where does this leave us? I want to try to the best of my ability to reach across the isle of my two worlds. I live as both the scientist and the Christian, and I have said \textit{both} sides of the argument above. I have Christians around me who don't understand what the scientific world is like, and even worse \textit{think} they know what it is like when in reality they have no clue. On the other side, I have atheist scientists around me who are quick to bring up all the failings of Christianity, but rarely attack the fundamentals that define what it means to follow Christ. They strive to convert not to understand. While I have a deep want for people to become Christians, I know not everyone will be convinced. I won't stop trying to convince those that are open to the discussion, but I would rather have people simply understand and respect the Christian worldview and then they can decide for themselves whether to follow it. Conversely, I want everyone to be a scientist. I know there will always be people that are hesitant of science, but I want them to have a real understanding of what is happening and not a twisted view portrayed by the media. I don't need people to see the same way I do, but I need everyone to respect the fact we all see the world differently.

I pray that I can try and bridge the gap. This book will not turn you into a scientist, but it will hopefully give a respect for the world I live in. This book also won't turn you into a follower of Christ. I hope that you can have a respect for my relationship with Christ and the value that has, but if the ideas presented do make you a follower, it won't be me who did the work.

So how can we bridge the gap? My belief is that life is a narrative. I will give you a look into the deep and fulfilling parts of both my scientific and Christian life, and maybe from there we can gain some insight into one another.

\part{Science is the Way}
\chapter{The Start of My Walk}
Prior to my undergraduate, I did the bare minimum in a lot of my life. You probably couldn't tell by my A's and B's on my report card, and relatively high-level courses, but for me it was the bare minimum. I was blessed with a mathematical mind, and I really enjoy learning. Getting A's in the highest math course was just what I did (even though I rarely did the homework). The clearest evidence of my lack of effort was my C- in Spanish (the only C I have ever received for a course in high school or college). I certainly didn't get that grade because the material was tough, and it wasn't because the teacher was bad. I earned that C- for the effort I put in. Not studying, not doing homework, and getting A's on tests works pretty well in high school for the classes I was already naturally good, and the rest I just got by. This got me all the way through my senior year of high school (which looking at my courses was essentially the average freshmen year of college). This was just who I was. I didn't take anything too seriously, and everything just worked itself out in the end. I got a score on the SAT\footnote{The Scholastic Aptitude Test (SAT) is the standardized test used for undergraduate college admissions. When I took it, it was already being phased out and replaced by the American College Testing (ACT).} that was the just above the score I needed to get a full scholarship for college. That was me. Just doing my bare minimum to stay on the upper side of the bell curve. It was nice. It was easy. It had little responsibility. It had little meaning.

My plans for college were to follow in my brother's footsteps (considering I was going into the exact degree program he went to, it was an easy decision). Hopefully finish up college in four years with no debt and get a cushy job as a software developer. From there I would probably work through the company for a while, and maybe come up with my own idea, strike it rich, and retire at 50. It was pretty straightforward. It was pretty easy. It had little challenge (considering all of this stuff had been easy so far). It had little meaning.

This entire time I was calling myself a Christian, and I believed it. I mean, I was going to church on Sunday \textit{and} Wednesday. I was playing in the youth group worship band and was on staff as a janitor. Some might say I was doing as much as I could to be involved in the church, but as soon as I left that building I was doing my same routines. I wasn't trying. I gave little effort to my relationship with Christ. I had little meaning in calling myself a Christian.

The summer of 2018 will forever be ingrained in my mind. I started my path to my `dream job' by getting a software developer internship at a hip company, but I lost my only friend I still had. In high school, my relationships with people were about as enriched as any other part of my life (meaning I was apathetic toward most of them). I had graduated high school, started college, and essentially was left with one friend. There was something about sitting and realizing I was getting my `dream job' but having no one there with me was tough. Don't get me wrong, my parents and siblings were (and are) always there for me, but it's a different kind of relationship. I tried to push the feelings of loneliness away and just focused on the upside, which lasted all of a few weeks. My `dream job' became the mindless weekday grind. The job lost its initial shine, and it wasn't exciting. Looking back I was stuck with some annoying grunt work for the first half of the internship, and it got a whole lot better during the second half, but when you have two parts of your life get crumpled up and thrown in the trash it isn't easy to continue. There were days that I wish I just didn't wake up. Maybe I was over dramatic, but I don't think I was. I was taking the easy route. I had little responsibility, and I had little meaning.

This was kinda when that bare minimum of going to church and bible study really came into my life. When you have little meaning in your life, you will be open to listening to any meaning people throw at you. It was this that really pushed me to devote my life to Christianity. I finally put myself as ``all in'' for something. Maybe at that point I just needed community, and a local game group would have sufficed. But going to bible study and talking about the real struggles of life was something that I wouldn't have necessarily gotten in a different setting.

By the end of the internship I was much more devoted to be a Christian, and to work harder in my life. It was when I thought I was starting to put my life back together that my father asked if I wanted to work in research. As someone who had industry programming experience, I had the ability to contribute in research in a way that the average second year computer science student likely couldn't. He told me I should leave my high paying internship (which they had offered me an internship to come back next year) and interview to work with the electrical nerve block group at Case Western Reserve University. Just when I thought I had everything together. I had the job. I was having meaning in my life. But I decided that I needed to push myself in every part of my life, and if I'm honest I wasn't being pushed as a software developer.

\section{Two Worlds Collide}
October 8th, 2018 was when I started my journey as a scientist. This was when I knew I was starting to push myself as hard as I could during my undergraduate degree. My first days of on-boarding I had to skip class to sit and listen to 401k plans I wasn't putting money into and all the facilities on campus I likely wouldn't take advantage of (our lab was at MetroHealth Medical Center so I was rarely on campus). This semester I was still taking a full course load, and now I was piling a part-time job on top of it (and I was still doing a few hours a week as a janitor). I can still remember sitting outside in the shade during a break between HR lectures and seeing a programming project being posted for my linear data structures course. I didn't take the easy route from this point on. I knew I had a cushy job as a software developer waiting for me whenever I was ready to stop (or at the very least a job scrubbing toilets), but deep down I had a drive for something more.

Going to work everyday, I would see the nameplates on the doors of the people that worked with me. Everyone's name ended with a PhD or PhD MD. Seeing those nameplates everyday, and knowing I had the least experience out of everyone, terrified me. From this, I pushed myself to work even harder. I read as many papers I could. I would work my 8 hour shifts without a break and would read papers and code from 9AM to 5PM. I felt I had 20 years of research to catch up on, and I needed to be caught up by next week. In hindsight, this was clearly an unrealistic expectation I put on myself, but I was on a path of life where I would push myself to the breaking point before I would take the easy way out. I genuinely believe it was this drive that got me through that semester. It was this drive that made me realize the potential I had within me, but I needed to believe in myself first. Looking at the pedigrees of those around me, it was hard to believe in myself (most of the time).

The next semester (Spring 2019), I took a graduate level computational neuroscience course at Case Western Reserve University as a second year community college student. This meant I was a second year undergrad in a room of graduate students and undergraduate seniors. I was the only non-senior undergraduate in the room. If looking at extra letters on people's titles intimidated me, then walking into that class was a completely different level of intimidating. There was only one student from my high school graduating class, that I know of, that was accepted into Case. To my knowledge, I was the only student in the room who wasn't accepted into the university through my high school academics or undergraduate academics. That left me again feeling like the child who stepped up to the major league batting box when at heart I wondered if I could even hit off a tee. I pressed on through my self-doubt and continued through the course. I stayed up late to work on homework and agonized over the material. But when we reached the final project of the class I had realized where I truly stood. On our projects we had to write a final paper and give a final presentation. The professor returned the final papers and said that anyone who wanted to could revise the paper and try and earn credit for points missed. When I realized I was the only person who handed my paper back, and said that I'll stick with my 95\%, I knew I had done something right on that paper. The second part of our final was to present out project to the class. As a computer nerd, I always knew I needed to work on my public speaking, so I worked long and hard to make a understandable presentation that conveyed a coherent thought. While my fellow classmates struggled through their presentations giving graphs that were too small to read, too confusing to understand, with little or no explanation to the importance of the topic, I felt I gave an excellent presentation. It certainly wasn't ground breaking, but I felt I finally realized that I had knowledge that I could contribute to other people, and that I could communicate it effectively. When I got an A in that class I finally believed in myself. Because that was my only class at Case, it left me with a 4.0 GPA on my transcript. That's something I'll be proud of. Not because it was easy, but because it was hard.

But why bother working hard? Staying up till 4 AM\footnote{This was the only time I actually stayed up essentially all night for a homework assignment. My main issue was that I kept running into an error because I forgot MATLAB arrays are 1 indexed not 0 indexed. If you know what that means, you understand my pain. But now I will never forget that fact...} to work on homework might earn a grade. That might help earn a degree, and it might even help me eventually earn a job to live out my life. But what is the point of all of it? Why work hard? Is it so that the next generation can have a slightly easier life? So that we can progressively usher in a utopia? The greatest ideal I have heard from a materialistic perspective is that we should strive to find a way to stop entropy. But supposing we learn how to sustain the universe \textit{ad infinitum}, what then?
\marginnote{Should I include how I got into graduate school here?}

\section{Christian for the Soul and Academic of the Mind}
The Christian life has always worked hand in hand with my academic mind. The Christian faith is not presupposed on a blind following of things that disobey reason. It's a viewing of the world that goes beyond what pure reason can give us. I don't think academia gives me a solid base to lay out answers to the moral landscape, and I don't think Christianity gives the knowledge to send a man to the moon. Both of them together make me who I am.

Hopefully going through my background has given academics some understanding of where I come from both spiritually and professionally. I will continue to defend reason, and defend the importance of academics in society. To state the importance of science, technology, engineering, and math would be an understatement.  They continue to help us in every single portion of every day. In my life, I wake up in a bed that took research and development to manufacture, to get up and stand on a carpet that went through the same process, to sit at my computer desk that went through the same process, to eat a meal that also went through the same process. To say science, technology, engineering, and math are important is a \textit{vast} understatement. In my own studies, I hope to add to the improvement of life around the world. I also seek to uphold the pursuit of knowledge that should characterize academia and scientific advancement. 

I also need to acknowledge that there are portions of my life that are best governed by the Word of God. Not because the laws of physics don't apply when I choose, but because there are some things beyond the measurements of the laboratory. When I give meaning to my life, I don't get it from a test tube. If I did get it from a test tube then I would have no reason to give any sort of drive. Why not take the easy route? Maybe an argument could be made that I need to achieve some sort of self-actualization, but it isn't clear to me that I absolutely need to give meaning to the big picture of my life. It isn't clear why I needed to assign my life some sort of meaning when I was at my lowest. If I took the materialistic route, I would come to the inevitable conclusion that in the end this is all meaningless (as King Solomon had already written in Ecclesiastes). 

\begin{center}
2 Vanity of vanities, says the Preacher,\\
vanity of vanities! All is vanity.\\
3 What does man gain by all the toil\\
at which he toils under the sun?\\
4 A generation goes, and a generation comes,\\
but the earth remains forever.\\
    
Ecclesiastes 1:2-4
\end{center}

So did I take the easy route by choosing God? Is my hard work in the physical and mental sense my compensation for a spiritual crutch? I don't think so. Hopefully throughout this book, you can see that my faith gives me the answers without merely being a spiritual crutch, and my work gives me a deeper understanding of the universe from which I continue to grow deeper in my belief of God.

There is no evidence I can ever lay out that will absolutely convince you in Christ Jesus being the Son of God and the Savior of the broken. There is absolutely nothing I could say if you haven't first opened your mind \textit{and heart} to the idea that it's possible. Some people may never think it is possible. Stephen Hawking, and Christopher Hitchens were two people who jump out to me as not having remained open to the idea. And someone like Richard Dawkins, who has stated that nothing would be able to change his mind. They have closed the door and let no one in. I urge you not to be so closed-hearted. I urge you to be open both to hearing the Word of God, but to also be willing to hear others challenge your understanding of the Word of God, or even question the validity of God. If either side closes their heart, then we have no discussion on the table.

\section{Understanding the Academic Worldview}
My hope is that throughout this book we can each see the benefits and pitfalls of our own thinking, and particularly the downfalls of not seeing the other side of the discussion. I hope that I can lay out a very clear reason as to why I am happy to be an academic, and why that doesn't inherently destroy my Christian beliefs. There are a lot of people that have become increasingly skeptical of the scientific world, and occasionally it's warranted. Occasionally, science can step outside of its bounds, but it isn't as often as some people want to think. We need to ensure that the Christian belief doesn't become synonymous with anti-science. How many more people do we need to put on house arrest for their scientific beliefs before we can agree let science explore the world without being beaten down by dogmatic and weak theology we use as a cover for our lack of understanding.

Academics and science in general has given us so much good, and I will repeat this numerous times. The simple fact is that the quality of life we have now is so far beyond what we had just two centuries ago, that its almost unfathomable. Technological innovation in healthcare, electronics, and automation have accomplished things we never thought possible. Not only has this allowed us to increase the average life span throughout the world, but given us the ability to communicate across the globe though an extremely complex network of satellites orbiting around the planet. To think we have come so far so fast is so mind boggling that we can't take the time to comprehend it. So much brain power is spent on improving technology that we have only increased the already breakneck pace at which we continue to grow. The achievements that science as a whole have given us are so vital to our everyday life that it's almost impossible to escape it. Even this book right now wouldn't have been possible a hundred years ago\footnote{Currently I'm writing this on a computer using \LaTeX, neither of which were in the realm of possibility 100 years ago}.

We as academics get to be at the forefront of this ever growing tsunami of information. It's an extremely intense job. While I may only have recently started, I have already seen my dad go through the same path. Living from one grant to the next, and working day in day out on a constant search for the next question that hasn't been asked. It's much closer to running your own business than something like a technician in an industry lab. What's interesting is that people would work so hard, for what can usually be so little. My initial pay as a graduate student put me just slightly above what I would get for a full-time minimum wage job in the DC area at the time. Why would I want to work more hours, in a high stress environment, to make less? It's because the pursuit of knowledge, and the search itself, is worth the price. I don't think non-academic Christians understand how much of an intense grind, beating your head against the wall from failed experiments, papers rejected, and grants not funded. The caliber of skills to be a scientist are not a simple 9 to 5. I hope that the scientists don't forget why they joined in the first place either. The drive to try and understand the next step in technological innovation can slowly be beaten down by mindless paperwork, but let us not forget that we still have the greatest job in the world.

Hopefully we can start to see that the academics of the world have clear pieces of information they contribute to humanity, and need to be taken seriously. They shouldn't be a group of people that we are terrified to listen to. We shouldn't be cowering in fear the next time a scientist finds that the earth isn't the center of the universe. And conversely scientists shouldn't abuse the authority that they have in society.

\section{Understanding the Christian Worldview}
Science has progressively tried to squeeze out any ``God of the gaps'' beliefs. It has dismantled the god of lightning, and the god of the tides because they weren't the God we needed. Those gods were place holders for our lack of scientific understanding. This leaves us with the question of how much of religion can we squeeze out. It seems at the current point we have a lot of moral questions and metaphysical portions of the human condition that do not appear to have been any more understood from scientific perspectives. I will try and bring up how far science has really come to squeeze out the poor theology that has plagued those who have claimed to have believed in God. But that doesn't mean science has squeezed God out entirely, or at least I haven't found the answers.

Belief in God in the Christian worldview is something I describe with a very particular word. I describe the Christian worldview, the Christian life, as ``personal''. Throughout this book I will try and take a look at how my faith in Christ has particularly impacted my individual life in a way that I don't think science could, or at the very least in a way science didn't. Christianity, as far as I can tell, is not meant to drive you directly to a strict set of 10 rules, but to drive to an overarching point. It's not meant in the same way an engineering textbook is meant to be understood. The problems of everyday life are not always solved with a calculator and a set number of rules, but with an overarching direction pointed somewhere. The question is where? I hope that the scientists can see that there are some serious questions, and Christianity has some excellent answers to those questions. 

I also hope that this book can reinforce Christians with where their bounds of belief belong. Regularly, we as Christians are quick to point out when science oversteps its bounds, but do we check our own bounds? If Christianity is really as good as some people claim it to be, then why aren't more people dropping everything to come and follow Christ? While certainly science has its own problems, Christians are nowhere near off the hook, and I would argue they have clearly a higher standard to uphold. Maybe when we look and criticize the spec in the academics eye, we can make sure we have pulled the plank out of our own.\footnote{3 Why do you see the speck that is in your brother's eye, but do not notice the log that is in your own eye? 4 Or how can you say to your brother, ‘Let me take the speck out of your eye,’ when there is the log in your own eye? 5 You hypocrite, first take the log out of your own eye, and then you will see clearly to take the speck out of your brother's eye. - Matthew 7:3-5}

So let's begin. How does science and Christianity form a compelling narrative in my life? You've already seen a little bit in particular life events, but what about the direction my life is going? Let's keep open minds, and open hearts, and see where that can lead us in reaching across the isle.

\chapter{On Being a Scientist}
\begin{figure}[h!]
\centering
\includegraphics[width=0.9\textwidth]{./figs/dunningkruger.jpg}
\caption{The Dunning-Kruger Effect shows that competence and confidence aren't equal \cite{Kruger1999}.}
\label{fig:dunningkruger}
\end{figure}
Being a scientist is quite literally the most humbling, and the most prestigious job I can imagine. Isaac Newton put it best that we see farther only by standing on the shoulders the the millennium before us. I think it's important to understand and reflect on the importance of our work, and the affect it has, and will have, on society. This should also give a chance for those who aren't scientists, or reject the the findings of science, to understand what the process really looks like.

In science we do the following:
\begin{enumerate}
\item Conjecture on the unknown
\item Test the unknown
\item Analyze the test results
\item Reevaluate conjecture based on analysis
\item Communicate
\item Repeat
\end{enumerate}


\section{From Conjecture to Reevaluation}
We all learned about the scientific method in school. It is the process from which we can generate knowledge of how the universe works. We start with an educated guess, and test if our guess is right. But there is something important in understanding these steps. They start with a conjecture. They start with the unknown, and spend time trying to gain an infinitesimal amount of knowledge in the grand scheme.

Why is it important to comment on conjecturing? I think it is something that is an important piece of our job that we need to really consider the significance of. Society relies on us to give initial comments on the unknown. That is a weight that can very easily be handled lackadaisically. This means we must comment intelligently on subjects, and remain objective.  

Testing and analyzing are the traditional steps associated with science and taught in school. It certainly is not an easy feat to create a good experimental protocol. More important than ensuring we can test a variable, is understanding the pitfalls of any given protocol. To understand the possible variance and points of failure within a experiment takes a lot of knowledge. Understanding where the possible unknowns can assert themselves in the measurement of an experiment means understanding the equipment being used, how the equipment can falter, how each piece of equipment can interact with every other piece of equipment, and how the layout of the experiment itself can have built in unknowns. I think the following quote really highlights the importance.

\begin{center}
``Since all models are wrong the scientist must be alert to what is importantly wrong. It is inappropriate to be concerned about mice when there are tigers abroad."

- George E. P. Box
\end{center}


As an example, we can look at some of the tests done for analyzing COVID-19 mitigation strategies  \cite{maskEffectiveness}. This particular study randomly assigned mask wearing to a various people across Denmark. How do you set that up to account for everything and give a complete assessment of everything? You can't. Until big brother watches every movement every citizen makes, we would never be able to make a perfect single analysis of mask wearing and COVID-19. Because we can't experiment in a perfect vacuum with a complete view of all experimental measures, we are left to understand what mistakes are built into our experiment. So what are some of the problems in the experimental protocol proposed for analyzing mask use? It didn't ensure that people actually wore their masks correctly. It didn't check whether people with masks are more or less cautious than people without their masks. There was no assessment of whether wearing a mask helps mitigate the spread when you are already infected. These are important shortcomings of the study. Every study has their own set of shortcomings, and they are an extremely important part of testing our conjectures. I would make the argument that the most important part of building an experimental methodology is to understand the shortcomings of that methodology. Without understanding the flaws of an experiment, the interpretation of the data to come will likely be misinformed and exceed the bounds of knowledge that were actually demonstrated.

Once we have tested, we analyze. This goes into a deep discussion on statistical analysis and data communication. As a data oriented person, I love looking at data and trying to visualize it in different ways. There are so many tactics with data analysis that it is easy to make a disingenuous data visualization. The hardest part is that it can be done inadvertently. It's not hard to forget to zero your axis values. It's not hard to create plots showing 10 different data series that completely confuse anyone who doesn't already understand the conclusion being made. Theoretically all of this is mitigated by an independent statistician, but regardless statistics can be used to misinform. We as scientists are usually communicating our results to fellow scientists, but occasionally (like the COVID-19 mask study) we are giving statistics to the public. Ensuring that we correctly calculate and interpret what it means to have a p value less than 0.005 or what the interpretation is when a study shows something to be statistically insignificant is an important part of the job.

Finally we come to reevaluating our conjecture, the last part of the standard scientific method. The hope is that our initial conjecture was correct, and we don't need to change our initial conjecture. Usually we find some nuance on the initial conjecture that might change, but sometimes we find results that are head scratching and require a large change in our conjecture. This hopefully also leads us back to more questions.

\section{Communication}
Now the scientific method generates our ideas, but I found that is only a portion of what it means to be a scientist. What good is it to generate knowledge if we don't communicate it. This can come with scientific papers, communicating directly with the public, giving presentations to fellow researchers, among many other ways. This isn't necessarily a part of the scientific method, but it's a crucial part of the scientific process. The problem with this portion is that we rarely focus on it when we think about the concept of science (or when we teach it in schools). What good is information if we can't adequately share it with people? This is much deeper than just making sure axis' on a plot are appropriately labeled, but questioning whether the axis' themselves are appropriate measurements for the data being shown. Stepping back and understanding the message behind the experiment is a nontrivial endeavor. Setting up experiments and collecting data is a straightforward process, but communicating why these particular pieces of data describe some larger phenomena is a much tougher process.

It was said that ``if you can't explain it simply, you don't know it well enough." The real challenge arises when you have focused on a question that is five layers deep into a field, and then you have to find the appropriate level within which to contextualize the question. I will use my first published paper as an example here. The title ``Combining Direct Current and Kilohertz Frequency Alternating Current to Mitigate Onset Activity during Electrical Nerve Block" brings us a series of questions to contextualize the findings. What do you define as mitigate? What is onset? What is electrical nerve block? What is the difference between direct current and kilohertz frequency alternating current? What is the applications of this research in the larger scope of the electrical nerve block field? What is the purpose of electrical nerve block as a whole? Each of these are completely valid questions. I didn't even begin to go into any questions about the modeling portion of the paper and the questions that go along with it (the computational modeling alone has textbooks upon textbooks written). Obviously we can't answer every single question in excruciating detail, so how much can we assume the reader knows? How much can we briefly cite previous work to avoid an extensive introduction? Not to mention we have a wealth of knowledge we would love to just cover pages in, but wouldn't create a cohesive thought. We had to put together a paper that gives a clear direction, and that is not a simple task.

I feel now is a good time to also give a short aside on the peer review process. My first manuscript I was on was accepted, like many, after revision. Being an undergrad, and having to respond to ``experts of the field" was nerve racking. I don't think I will ever fully consider myself an expert in the field of electrical nerve block (maybe that is just the impostor syndrome), but I didn't feel like I was prepared. I was expecting them to tear down my work, point out the flaws that I hadn't seen, and we would have to rewrite the manuscript and leave out the modeling portion I had worked on. To my genuine shock, the comments were light, some criticisms bigger than others, but this became one of the moments I felt like an expert. I didn't do exactly what their criticism said we should change, and I stand by my work, and must have given a convincing argument why my ideas were valid and didn't need to entirely be changed. Somehow they accepted those changes. I had someone who thought I should do something a certain way, and I went and did it another way, and they agreed with my way of doing it. That moment when you have stood up for your ideas, and they aren't shot down. I think those are the moments where you feel like a researcher. When out of the blazing fire, you still have an idea that hasn't been destroyed in the heat. In that moment I think I started finally moving up on the confidence scale of the Dunning-Kruger Effect (Fig.  \ref{fig:dunningkruger}).

\section{Repeat}
This step is something that has somewhat struck me. There is the obvious repetition of randomized experiments to gain a large enough data set to give some sort of statistically significant result, but what about the repetition of the scientific process itself. The main note that I find somewhat mysterious is the level of creativity required in this endeavor. Numerous experiments become the self-explanatory extension of previous work. When you answer one question, three very obvious questions arise. Setting up the initial work for a new question can be creative, but the following work requires relatively little creativity. In some respects, we don't necessarily want every paper to be extravagantly creative. It's hard to determine the worth of exciting new research that is exceptionally creative. Progressive research, building off the shoulders of giants, gives us a constant direction.

Does this mean that research isn't creative? Absolutely not! While the questions that arise give us clear direction, the method with which we solve those questions gives us an excellent sandbox to confine the creativity into a useful direction. Occasionally someone comes along and does something so outlandish it flips an entire field on its head, or starts a completely new one. But this usually isn't the case. Not everyone is an Albert Einstein or a Stephen Hawking. And in some ways, not having everyone revolutionizing their respective fields constantly keeps the scientific progress stable and directed. 

\section{What is a scientist?}
The previous few pages have gone over the general job description of a scientist, but what does it really mean to live my life as a scientist. I think it boils down to the following. I ask questions, and when needed answer previous questions. The current questions don't have answers yet (if they did, then I wouldn't keep questioning them). Previous questions we might have used as a springboard to find us in new uncharted territory. When in the new territory I can ask more questions, and look back at the previous work and give some answers to old questions.

In some ways, we are all scientists IF we ask questions. If you walk around knowing all the answers, but not asking questions, then you aren't a scientist. There are plenty of people in this world that are not scientists. The questions then arises, how much do we each live our own lives as scientists? The academics and Christians alike have sometimes ceased to ask questions to the opposite sides. I think an important part of co-existing on this planet is to openly ask question to understand the other viewpoints. We might have conjectures (or as some of us call them ``answers") to questions, but are we openly searching for other answers. Have we closed off the walls to an open discussion? Even if we maintain our own preconceived notions with our feet planted firmly, do we still let others give them a little push from time to time? An important part of being a scientist has been holding my own ideas close, but not keeping them from seeing the light of day. We don't want to live in a complete silo, nor should we live our lives that way.

We are then left to ask if we are spiritual scientists? Do we live a spiritual life that is built only on answers, or on questions. I would say I have some data that leads me to a particular conclusion, but I certainly don't stop asking questions. I also, as a Christian, don't say that I know all of the answers, but I believe in someone who does have the answers. The atheistic/anti-theistic academics seem to be just as closed off as they claim Christians to be. They are often unwilling and closed to questions, and likely its due to the widespread often unwilling closed Christians that exist. To find a way to have these discussions, we all need to be spiritual scientists. We have preconceived notions, but more importantly we have questions and a willingness to search for answers. Hopefully we ask questions, and when needed answer previous ones.

\chapter{The Order of the Engineer}

\section{A need to be more}
Throughout my undergraduate degree, some of the darkest engineering disasters were brought up regularly as a reminded of what might be if you don't step up to the challenge. Here are just a few of those events discussed. They will forever be a reminder of what might happen.

\subsection{January 28, 1986}
On a cold Florida morning at Kennedy Space Center Launch Complex 39, a space shuttle launch as part of  NASA mission STS-51-L was under way. The temperature on the site was measured to have dropped to -8 degrees Celsius over night and was approximately 2 degrees the morning of the launch. Additionally, due to the large tanks of supercooled liquid hidrogen and liquid oxygen, the wind created pockets of air around the base of the shuttle that were measured at -13 degrees Celsius.

Upon ignition an o-ring partial failure can be seen by launch pad cameras showing black smoke emerging from the aft field joint on the right strut brace roughly half a second after booster ignition at 11:38:00 AM. Another eight puffs of smoker were observed from 0.836 through 2.5 seconds into the mission indicating further erosion of the o-rings. 56 seconds after launch, Challenger passed through the worst recorded wind shear in Shuttle program history. At this point a reduction in chamber pressure was seen with the appearance of a small flame from the aft field joint. 72 seconds after launch the lower strut finally gave out. 1 second later white vapor of liquid hydrogen was leaking out and causing surges in acceleration. The acceleration caused the hydrogen tank to come into contact with the oxygen tank and a violent burn began erupting the whole vehicle in flames.

Francis R. Scobee, Commander; Michael J. Smith, Pilot; Ronald McNair, Mission Specialist; Ellison Onizuka, Mission Specialist; Judith Resnik, Mission Specialst; Gregory Jarvis, Payload Specialist; and Christa McAuliffe, Payload Specialist, and Teacher; all died in the fire. The next week the President at the time Ronald Reagan concluded his 1986 State of the Union Address with, ``We will never forget them, nor the last time we saw them, this morning, as they prepared for their journey and waved goodbye and `slipped the surly bonds of Earth' to `touch the face of God.' ''

\subsection{July 17, 1981}
At 7:05PM Central Daylight Time in Kansas City, Missouri, guests at the Hyatt Regency Hotel heard large pops from the fourth-floor walkway. The pops were paired with a few inch drop of the fourth floor walkway before the entire walkway gave way and fell down on top of the second floor bridge which also subsequently collapsed under the additional weight.

At the time there were approximately 40 people on the second floor bridge, 20 on the fourth floor walkway, and approximately 1,600 people gathered in the atrium below for an event. The collapse caused 114 deaths and 216 non-fatal injuries, and the rescue operation lasted 14 hours. This event remains the deadliest non-deliberate structural failure in American history. See Appendix \ref{appendix:hyatt} for a list of 105 names of those lost.

The cause of the accident was a change during the construction of the walkway in which two vertical rods supporting the walkway were placed through the horizontal support beam on a welded joint as oppose to one single rod all the way from the second to fourth floor. This poor design caused the rod and nut to be pulled up through the welding joint under load, causing a popping sound and few inch drop before the nut would be caught on the opposing side of the beam, but would then (a few seconds later) rip through the opposite side of the beam completely coming out and dropping the entire weight of the walkway. Jack D. Gillum, the designer of the original design, stated that ``Any first-year engineering student could figure it out."

114 deaths and 216 injuries, over a \$100 million dollars in settlements paid, all because of a relatively trivial design flaw.

\subsection{August 29, 1907}
Across the St. Lawrence River was a bridge in the process of being built. This bridge would become the largest cantilever bridge in the world upon completion and would be a feat of engineering. Previous workers had noted that prior to this day, and in the previous months, some of the members of the structure appeared to have been bending (and deflection measurements had been taken to show this), but it was passed as an error in material manufacturing and that the structure was believed to remain stable. Earlier in the day, Theodore Cooper finally agreed that the issue was serious and sent a telegraph from New York to the Phoenix Bridge Company (who were hired to design and construct the bridge) that they should ``add no more load to the bridge till after due consideration of facts."

That message was not given to the construction before that afternoon when the south arm of the bridge collapsed into the river. At the time of the collapse, 86 workers were on the bridge, and 75 of those were killed in the collapse, leaving 11 injured. Throughout the aftermath, the Royal Commission began analyzing the disaster to determine the root cause. Of their 15 conclusions, 3 of them give a deep look at the fundamental problems.

\begin{center}
``(c) The design of the chords that failed was made by Mr. P.L. Szlapka, the designing engineer of the Phoenix Bridge Company

(d) This design was examined and officially approved by Mr. Theodore Cooper, consulting engineer of the Quebec Bridge and Railway Company.

(e) The failure cannot be attributed directly to any cause other than errors in judgment on the part of these two engineers."

Royal Commission Report
\end{center}

\section{Willingness to not do right}
The previous three disasters had known issues (in the case of the Challenger O-rings), or had poor planning (in the case of the Hyatt walkway and Quebec Bridge). When human life is on the line, the level of detail needed is much higher than some trivial piece of entertainment software. But how much do we think about the ramifications of what we make?

As a scientist (and engineer) technical mistakes can lead to some serious problems. What about the AI chatbot that learned to spew out racist comments? What about writing some code wrong that means my dataset isn't correct? What happens when my conclusions are wrong? What happens when I publish incorrect or misleading data? Every step of the process builds, and it's important that I get everything right (or at least as right as I can).

But what about in general? I'm not launching a space shuttle into space space on a daily basis. Not everyday are you building or planning out the world's largest cantilever bridge. But everyday I make decisions. Everyday I have interactions with other people. Let's take a look at what we should do.

\section{Order of the Engineer}
Upon graduation of my engineering degree, I was invited to join the Order of the Engineer. As part of the organization, there are no yearly meetings, and no conferences. The only thing we have, is an induction ceremony and an obligation.

\begin{center}
I am an Engineer. In my profession I take deep pride. To it I owe solemn obligations. As an Engineer, I pledge to practice integrity and fair dealing, tolerance and respect; and to uphold devotion to the standards and the dignity of my profession, conscious always that my skill carries with it the obligation to serve humanity by making the best use of the Earths precious wealth. As an Engineer, I shall participate in none but honest enterprises. When needed, my skill and knowledge shall be given without reservation for the public good. In the performance of duty and in fidelity to my profession, I shall give my utmost.

Obligation of the Engineer
\end{center}

This is an obligation to do everything in my profession with the highest standards I can uphold. A need to double check that calculation I was unsure about. A need to check whether the construction of a design will be sufficient for the project. A need to check whether my code actually does what I have claimed it to do. A need to be more than the bare minimum.

The Order of the Engineer originated from the Canadian ``Ritual of the Calling of an Engineer" but due to copyright reasons that name did not carry over the border. But the thought of this ideal came from the Quebec Bridge collapse. An entirely avoidable collapse, due solely to the failure of the engineers. And when failures cause death, we need to really look back at how the previously thought insignificant actions can become the most significant decision of someone else's life (or in the case of the Quebec Bridge, the most important decision for 75 lives).

I mentioned that there are no meetings, but there is one other thing that signifies being a part of the Order of the Engineer. Upon taking the pledge every member is given a ring. The ring is meant to be placed on the engineers working hand pinky finger. Notice that it's referred to as the working hand. The idea behind it is that every time the engineer puts their hand to the table to work they have a constant reminder of their obligation. What would the NASA meeting room have been like that morning if everyone was reminded of the magnitude of the situation and the need to hold a higher standard?

\section{Being More}
What could we be if we did as much as we could? Think about what the world would be if every single person did everything they could do to make the world a better place? The probability of this happening is extremely low, so let's ask a different question (a better question). What would your life be like if you genuinely stepped up to the plate of life to do the best you could? Not just slump around from one random job to the next, just doing a mindless grind. What about putting in that extra hour to really take your project to the next level? What about taking that coworker out to lunch because you know they've been having a bad week? What about calling that family member you haven't talked to in a while?

For me I had to challenge myself. I couldn't just settle being \textit{enough}. I knew I could get by in school doing the bare minimum. I knew I could get a job and keep it doing the bare minimum. And in all honesty, I could get through life doing the bare minimum. It's easy. Its got very little to worry about. But it doesn't have meaning. Life is meant to be something more. To give all that you can. To do all that you can, because what else are you going to do? You aren't going to sit on your death bed wishing you did less. Be more. Be the best person you can be. Just imagine if everyone tried to do everything they could to the best of their ability. How different of a world would we be in. How can I in a tangible real way start being more?

Life is a story, not a set of equations, and you get to help write it. Do you want the pages to be a boring grocery list of accomplishments, or are they going to be in depth relationships and connections with people. 

\begin{center}
14 As obedient children, do not conform to the evil desires you had when you lived in ignorance. 15 But just as he who called you is holy, so be holy in all you do; 16 for it is written: Be holy, because I am holy.

1 Peter 1:14-16
\end{center}

\part{Science isn't Everything}
\chapter{Data}
My academic life has focused a lot around computing. My Bachelor's degree is in computer science and engineering. My undergraduate research was in computational neuroscience. And my PhD is in computer engineering. Together technology has vastly changed how the world looks at a rapid pace, and I have seen interesting parallels in how we view the technological world, and how we can view our spiritual world.

As a scientist I revolve my life around data. Analyzing and and generating it. Each scientist is always creating new datasets (or analyzing old ones) to create new pieces of knowledge for us to understand the universe. How does this portion of the scientific method inform us about the world?

The world is generating data at a breakneck pace never before seen in the entire previous 20 centuries combined. Data comes in many forms, whether it is JPG, GIF, MP4, or for a scientist DAT, CSV, or XLSX formats. The data of this world is the most important asset we have. In top tech companies, their greatest asset is not their code, its the data they have. Google has the value it does because of its data. Are we also a sum of our data? And how should we value that data?

\section{What makes data valuable?}
Having spent time studying computer science and engineering as an undergrad, I can go through and explain from bottom up how data is stored on a machine, and from there maybe we can see where this can translate to how we can look at ourselves. We can take a look at one of the servers running in the massive data centers around the planet that hold the most valuable item on the planet, \textit{our} data. We all have heard of the 1's and 0's that computers use, and that's a good starting point for analyzing data. The 1's and 0's are stored in multiple locations across a machine. In a hard-drive or solid-state-drive the 1's and 0's are non volatile (they don't disappear when power is turned off), and then copied and stored in volatile memory in RAM and the CPU cache for faster access as needed. These 1's and 0's are typically a positive voltage stored across some capacitor or stored by magnetic polarity on a spinning platter. If I took a random drive from Google, and disassembled each component to the individual bits, how much would each bit be worth (a single 1 or 0)? What about a byte (8 bits)? What about a kilobyte? What about the whole drive? What about the entirety of the data saved on all the servers in the data center? When does the bit become valuable?

I doubt anyone is going to give me any money for a single bit cut from a hard-drive platter (maybe I would get a few thousandths of a penny for the metal itself). We can say pretty clearly that the bit level is typically not what we consider valuable. So what about the byte value? To analyze this, we should use a concrete example.

To securely store a password, we don't actually store the password. What we should store is some scrambled version of the password, and just like a scrambled egg, once your password is scrambled it can't be unscrambled (or to be more precise we say it is computationally infeasible to unscramble the password). Having this scrambled version means we have a piece of data that is complete gibberish, but it is a specific piece of gibberish. We can take another input and scramble it the exact same way and check if it matches our stored scrambled password. Because no two passwords will scramble the same then we know that the password entered had to match the password that was scrambled to create our saved data. This process of scrambling is known as hashing a password, and the details aren't as important as knowing we have pieces of data that are psuedorandom, but have a lot of value. How much is your bank account password worth? How much is your email password worth? I would guess those have become the two most valuable pieces of information in a persons life (in terms of monetary value at least). If I only had half of your hashed password, how much is that worth? It's worth nothing. I can't use half of a hash, I need the entire unscrambled password to have any value. How much is the entire hash worth? It too is worth almost nothing. A sufficiently scrambled password means the only way to figure out the password is to guess and find hashes that match. A good hashing function (currently SHA256) requires a lot of guesses before figuring out what password hashes to the same scrambled mess (enough guesses that it would take billions of years to find a match, or in other words it is computationally infeasible).

Further, we can argue the hash on its own is worth nothing, because it would be meaningless. You would likely have no idea what it is, and even if you could hit matching password on the lottery of astronomical scale, it is still meaningless. I never said what website the password is for (and the username that goes with it). Handing someone a string of 1's and 0's is meaningless, unless I tell you its purpose. Data has its value because it has a purpose. The purpose of my hashed banking password is to make sure that when someone logs into my account, that someone is always me. We have to abstract away from the physical data (the capacitors and magnetic platters) to determine the value of data.

\section{What is the value of our data?}
To ask what our data tells us, we have to ask what is our data? Our data comes in many different forms. We can look at each person psychologically (enneagram, big five, or whatever your preferred metrics are). We could look at intelligence scores (general IQ or topic specific). We could look at DNA for indicators on the micro level. Finally, we could look at your personal experiences throughout your life. Each of these data points give a particular piece of information. From a material point of view we should theoretically be able to combine your DNA and personal experience to determine everything about you. Similar to a system of equations that governs your life. If we know the equations, and we know the initial conditions, then we should be able to determine every single point along the path. 

This leads to the extremely important question, what is the value of our data? This has come into question in multiple different portions of life. In a practical way, we see data being something to use (or abuse) for insurance. As a male, the data says I'm more likely to get into an accident than my female counterpart, so my price of insurance goes up (and we have accepted this, because that's how the math works).  This works well for car insurance when we are mainly worried about making sure our vehicles are covered, but what about health and life insurance? DNA has the power to completely change the way we handle these kinds of insurance. We can pull out plenty of health predictions from DNA (and we are only finding more), and there seems to be a bit of backlash against this. Should the person with a higher probability of cancer pay more in health insurance? From a purely mathematical standpoint, probably. The more you cost the insurance company, the more you should pay. This doesn't mean that you shouldn't have the opportunity for healthcare, but that your insurance is simply going to be more expensive than the someone without the predisposition.

This is the much tougher question, what do we do with people who are deemed ``inferior" on any metric? I really don't take this subject lightly. I grew up with a mother and sister who taught students with developmental disabilities, behavioral disabilities, and other cognitive disabilities, and a father who worked on improving the lives of those with physical disabilities. I was brought up with the fundamental belief that all humans are valued because they are all individually made in the image of God. But if I were purely scientific and only valued the metrics of a person, then I could rank people by their potential value. Clearly a Bill Gates, Mark Zuckerberg, Mother Theresa, or Ghandi all have higher value than just any random truck driver right? I'm not to downplay truck drivers, I simply use them as an example because transportation is the largest job category in the US. We clearly didn't like when Hitler valued the people of particular race as more valuable than other races, so do we value all humans equally? And if so, why? I realize that for many people this is a crazy question, it is a \textit{given} for most of us that all humans are valued, but we need to juxtapose that with the metrics of humanity. If we are only the sum of our data, then we must come to the conclusion that some humans can be valued more than others. 

We could come with a few options from this. We can say some life is purely inferior to the point of being worthless or even harmful to the universe. This could be to the extent where we would be better off eradicating the inferior people. It seems that the overwhelming majority do not believe this, and hopefully the events of the 20th century said enough about the idea of eradicating any group of people. The next option is that we think some groups are just inferior to other groups, and I think a lot of people implicitly agree to this (maybe it is reasonable to say the founders of the Bill and Melinda Gates Foundation are superior people than the average serial killer in prison). It makes it hard for me to see why people draw the line, if they are inferior then do we really look at the ramifications of that? What if our genes are inferior? Maybe we just should set our bars lower and not try to change too much in the world. What if our genes are so inferior that we know we can't provide anything, and will, in mathematical terms, be a negative value on society. When we start ranking value of people at any level, that leaves someone at the bottom. We could argue that everyone has some positive value, just some have a more positive value than others, but does that mean when someone murders Bill Gates that we give a worse sentence? Is that right? If we kill that person at the very bottom of the value ranking, do we give a lesser sentence for killing them? Can we create a ranking system of everyone on the planet so we can give out the world's resources appropriately? When we usher in the utopia we so wish for, how will we value humans? We say all humans are equal, but are some of us more equal than others?

\section{Sum of our data}
The last option is that all humans are valued equally regardless of anything. Maybe we believe this at some level, but I don't think we believe this at a general level. I think it's not a common view to say that as a human being, Adolf Hitler is just as valued as Mother Theresa. Not in terms of their affect on society, but that they are both fundamentally humans of equal value at their core being. I can't settle in my heart the way we can value people differently. I can't sit well with the concept that some people are more valued or less valued. 

\begin{center}
I was just guessing at numbers and figures\\
Pulling your puzzles apart\\
Questions of science, science and progress\\
Do not speak as loud as my heart

The Scientist - Coldplay
\end{center}

Maybe alt rock lyrics don't speak to you as much as they may speak to me, but I think it has a lot of truth in it. We can look at the puzzle pieces of our metrics. We can look at all the numbers that make up our existence, but that doesn't speak to something deeper. I don't think I can value the starving child in a third world country with no education as less valuable than the average high school student in the US. Clearly the high school student in the US will have better education and has a statistically higher chance of improving the world as a whole, and by looking at all the metrics would likely be considered the ``move valuable." But it can't sit well in my heart that someone is less valuable than others. Maybe they contribute less to society, but if our contributions are our only value then we are going to have a very unequal society when we bring in the utopia. 

Maybe another quote might provide a bit more pungency then 2000's teen angst lyrics, and should really ring true at the heart of the problem of valuing people.

\begin{center}
``If we present a man with a concept of man which is not true, we may well corrupt him. When we present man as an automaton of reflexes, as a mind-machine, as a bundle of instincts, as a pawn of drives and reactions, as a mere product of instinct, heredity and environment, we feed the nihilism to which modern man is, in any case, prone.

I became acquainted with the last stage of that corruption in my second concentration camp, Auschwitz. The gas chambers of Auschwitz were the ultimate consequence of the theory that man is nothing but the product of heredity and environment; or as the Nazi liked to say, ‘of Blood and Soil.’ I am absolutely convinced that the gas chambers of Auschwitz, Treblinka, and Maidanek were ultimately prepared not in some Ministry or other in Berlin, but rather at the desks and lecture halls of nihilistic scientists and philosophers.” 

- Viktor E. Frankl
\end{center}

Those of us who claim to be scientists, we need to really look at that because it is a bold claim. We are the ones that develop the metrics that people can use to value people. Yes, they certainly help us understand ourselves, but we need to contemplate the ethical and philosophical ramification that can arise from these metrics.

\begin{center}
``Scientific progress makes moral progress a necessity; for if man's power is increased, the checks that restrain him from abusing it must be strengthened." 

- Madame de Stael
\end{center}

As we create more metrics to measure people, what shall we do with them? It doesn't sit well if we keep measuring someone's value as a human being with these metrics. We need to come to an agreement that everyone is of equal value as a being even if they don't produce equal value in society. Maybe there is still an argument about what should happen to insurance prices as we continue to produce more data, but there shouldn't need to be a discussion of whether a fetus with three copies of chromosome 21 is less valuable than a fetus with the typical two copies.

This leaves us with the real fundamental question. If we have equal value, why? If everything material is saying that value is quantitative and/or relative, then where does our core value come from? Is it possible that maybe our value is something not in the material? Maybe your value is similar to the hashed password, only valuable when the creator of it tells us why.

\chapter{The Use of Models}
\begin{center}
``Since all models are wrong the scientist cannot obtain a ``correct" one by excessive elaboration. On the contrary following William of Occam he should seek an economical description of natural phenomena. Just as the ability to devise simple but evocative models is the signature of the great scientist so overelaboration and overparameterization is often the mark of mediocrity."

- George E. P. Box
\end{center}

When I started working at Case Western Reserve University, I was hired as a modeler. At the time, I thought it was the essentially the same thing as programmer, but over the years it has taken on a different (and more accurate) meaning to me. I think all of us are modelers in the fullest sense of the word, and my job just happened to be with a very specific model. But we all work with the most important model, our individualized model of the universe. We each see the universe uniquely, and we each interpret phenomena differently, but we must ask, at its heart, what does it mean to be a modeler? Is this a part of science that is only for the select few in academia, or is this something that we all do in our lives?

As a Research Technician at Case, I spent all of my time sitting in front of a computer,  reading papers, writing code, and running simulations (or as I liked to call them experiments). I was lucky enough to start with someone else's code base and start running simulations immediately, but that doesn't mean I knew what I was doing. Understanding the model required understanding how the model was created, what it was meant to be used for, and how we were using it. The particular modeling I was doing focused on electrical stimulation of axons to block action potentials (which is then applied more generally to blocking nerve activity). You certainly don't need to be a neuroscientist to understand modeling, but I think it is important for us to look at a particular model, and then work our way outwards to the more general modeling we do in the world.  

I would spend my days changing the stimulation parameters (sending more current to the axon) or changing the axon parameters (like changing the axon size), but what value does a computer simulation give to researchers? Some of you might think that there are obvious benefits, and there certainly are some obvious ones, but there are some more nuanced benefits and disadvantages to using a model. The two most obvious benefits are that I can run simulations in an automated fashion 24 hours a day, 7 days a week, 365 days a year, and I don't even need to watch them. Regularly, I would setup a simulation on my computer, and just let it run, and I would go do something else. My \textit{in-vivo} counterparts had to be awake and alert the entire time during their experiments. I ran all of my simulations on only 2 computers (totaling to a cost of ~3000USD) while my counterparts had thousands of dollars worth of equipment and used a sprague dawley rat every experiment (model number SD-F or SD-M). The precision of my machine can go to whatever level I need, and certainly they can't get nanoAmp precision with a stimulator running at 1mA like I can. It's pretty safe to say modeling will always have it's place in science (and has always had it's place in science).

Modeling is how we conduct science. This doesn't always mean that our models are run as simulations (though that may be the common use), but they are much more than that. We can go back to 2002 when the MacIntyre Richardson Grill (MRG) model of the myelinated axon was created \cite{McIntyre2002}. That set of equations used in conjunction with implicit differentiation methods allowed for modeling the electrical impulses of an axon. Going back further, we could go to 1952 when Hodgkin and Huxley created their famous HH model (also named after themselves) which modeled unmyelinated giant squid axons \cite{Huxley1952}. But certainly models go much further back! All the way back in the 17th century we can look at how Galileo modeled gravity. Going back further still, we could see how, in around 400BC, the models of the universe were created. Going back to the first known mathematical theory we can see how the Babylonians and other civilizations were creating models of triangles (with what are now known as Pythagorean triples). All of this gives us a rich understanding that the entire history of science has been about creating models. Those could be mathematical models (like gravity or Pythagorean triples), those could also be mathematical models used specifically in a computational setting (like the MRG model), or they could even be biological models (like the sprague dawley rats used in every one of our \textit{in-vivo} experiments). 

When I had learned that the rats we use are in themselves a model, I was thoroughly confused. At the time I thought that I worked with a model (in a computer) but they get to work with the real ground truth. But science has never been about assuming we work with a ground truth, because in science how can we define the ground truth? The problem with science is that we don't know what we don't know. That is a HUGE problem. When I was working on electrical stimulation, we didn't fully understand everything that was happening to cause the phenomena of electrical nerve block. From this lack of understanding we would build out a model and see if our model recreated the results we hoped. That is science. Creating a model, and seeing if the model can recreate your results. There is a constant cycle of trying to approximate the unknown with a model, and then trying to make sure the approximation accurately recreates what was being modeled in the first place.

Typically when creating models we will follow Occam's razor. And this is how we would want to conduct science. I could create a model of the universe from looking inside a glass bowl (as mentioned by Stephen Hawking in The Grand Design \cite{hawking_mlodinow_2010}), but this would be an exceptionally cumbersome set of equations to wrap my head around. Instead, we look for the simplest model that accurately recreates the results. For our electrical stimulation, we could simulate conduction block by only looking at currents flowing across the membrane. Does this mean we can confidently say that ionic currents likely play some role in electrical nerve block? Yes, (and considering electrical impulses travel along an axon by utilizing ionic currents this makes sense). Does this mean we can confidently say that ionic currents are the main or only component of electrical nerve block? No. There were plenty of other phenomena that our model did not incorporate. Our model didn't include ionic accumulation, metabolic reactions, electric field movements of charge, possible magnetic field movements of charge, and the list could go on. What would it take for us to create a ``perfect" model? 

\section{The Perfect Model}
I am certainly no physicist, but I have tried to learn what current physicists are using to try and disprove the existence of God, or prove the ability of the universe to exist without a Creator. The question is posed on both sides to state their case, and in this context, present their models, and see which model stands above the rest. 

A good model is one that accurately recreates phenomena. Our computational model of a myelinated axon (among many other models) was able to accurately recreate electrical signals propagating along the nerve. That was the original intent of the model, to recreate the phenomena that sends signals throughout complex life (the humble action potential). A great model is one that accurately recreates phenomena not originally specified in the creation of the model. Going back to our axon model, it could recreate the phenomena of electrical nerve block. When a model goes beyond its original intentions and still recreates phenomena, then we know a model is great. But what does it mean for a model to be perfect?

The perfect model needs to incorporate all phenomena, both known and unknown. I'm hesitant to say whether this will ever be possible. With my understanding of quantum uncertainty, this model will never be able to tell us exactly how the universe reveals itself, but will only give us a probabilistic representation. Regardless, I am unsure whether it will ever happen. The Godel incompleteness theorem seems to be interpreted that we won't be able to accurately create any model to explain everything (and maybe that's great job security for me), but that certainly doesn't mean we will stop trying (nor should we). We also would theoretically be able to potentially create multiple perfect models by modification, thus creating multiple equally valid models (even though Occam might have something to say about the preferred model). A model may predict that something plays a particular role (similar to how we looked at ionic currents playing a role in electrical nerve block), but we cannot say that any particular model is the only one that works. For example, we do not need to break down our model of physics to quarks because we can't isolate quarks. This means we could have two models, one that breaks down to the level of quarks and one that doesn't, and both are valid. Neither of these two models explains why either is there in the first place, and why we can build a model that can recreate anything. The bigger question is then why would any model actually work? Why is the universe able to sustain such laws, and not be pure chaos of random particles spread across more or less than 11 dimensions?

The problem of saying that a model from science is inherently better than a model from religion is that the model from science is lacking in a fundamental area. The model of the universe will only ever be attempting to recreate what is seen, but it doesn't give the reasoning for why it is seen, and cannot express anything on what is not seen. This is where the model of science stops and the model of religion can begin (notice they are not competing, they are complimentary). The first point of the Christian model is the idea that it is self-describing. ``In the beginning God..." the self-describing nature of God is foundational. That is what makes God, God. He is the ``was, and is, and is yet to come". It is an all encompassing concept, and put into a concrete form in His very nature.  From this we get that the universe is given form from its formless and empty shape. Does this mean that we don't need a model that science produces, and that the model of God is all we need?

We could then discuss the models as venn diagrams that intersect in some spots, but it might be better for us to think about the model of God as a lens. He is an all encompassing idea that can give us meaning and direction to the model from science. It isn't about filling in the gaps that science can't answer, but giving an entire worldview from which we have science that is meaningful.

\section{The Place of the Religious Model}
How do we hold two different worldviews together? How can we conduct experiments based on the idea that our observations exist in a finite universe that can be modeled while still believing in an omnipotent creator that can at any moment manipulate the universe in such a way that it could break our model at any observation? Instead of focusing immediately on all the ways the two seem contradictory, let's take a moment to look deeply at what all the religious model has offered me, and then we can come back and integrate them to a deeper (and more satisfying) conclusion.

Living with the belief that I was created gives me a satisfaction in my soul. It's a deep satisfaction that I can't find anywhere else. On a personal note, as the awkward computer science nerd that I am, self-confidence and self-value did not come easy to me. Rarely do I feel I can love the person that is in this body, but knowing that I am created, and that the creator of the universe has given me a purpose gives me the ability to care about me.

\begin{center}
13 For you created my inmost being;
    you knit me together in my mother’s womb.\\
14 I praise you because I am fearfully and wonderfully made;
    your works are wonderful,\\
    I know that full well.

Psalm 139:13-14
\end{center}

C.S. Lewis had a similar line of reasoning that our innate want and value of a Creator is evidence for it, and it adds the personal and arguably human touch that our lives desperately crave. Knowing that I am loved by a creator is something that gives me reason. It is from that love that I can build my reason for life, and its the way I can stand up in the mirror and love the person who stares back at me. This sinful conglomeration of self-pity is worth something. 

\begin{center}
High upon a lonely ledge\\
A figure teeters near the edge\\
And jeering crowds collect below\\
To egg him on with, ``Go, man, go!"\\
But who will ask what led him\\
To his private day of doom\\
And who will answer?\\
\medskip 
If the soul is darkened\\
By a fear it cannot name\\
If the mind is baffled\\
When the rules don't fit the game\\
Who will answer?\\
Who will answer?\\
Who will answer?\\
Who will answer - Ed Ames
\end{center}

I don't know if I would be able to find my meaning to life without having someone first say I am valuable. I was blessed to have parents that loved me unconditionally but, at least for me, when I was left to my own thoughts I had no love for myself. I can't imagine if I would have ever been able to just find my own path. The view that we are a random conglomeration of cells doesn't give me the ability to look myself in the mirror. Even considering the idea that it was hundreds of thousands of years of evolutionary refinement to make me would only give me a greater sense that I myself am a speck of nothing. Yes we are all capable of change, but what would the point of change be? So that the next speck of nothing can focus less on needing food and shelter and more on their own insignificance. It is such an overwhelming concept for me to consider us nothing more than random chance, that I could only see myself living in nihilism. Maybe I am just part of the group that tends toward a lower amount of self-worth and that would put me in a higher risk category of nihilism, and that too is just random chance of gene mutations. But this whole idea the material is everything doesn't sit well with me. Why would a universe form something that would want to harm itself? That seems like the ultimate failure of natural selection, and maybe I should have just let natural selection run its course.

The material world view left me three options. The first and least likely was that I would harm myself or something worse. Personally, I didn't feel the need to cause pain to myself, and would have rather just simply passed peacefully in my sleep. Since I certainly didn't have the physical or emotional strength to go through with an idea like that. The second option I'd that I likely would have spent my life doing what I thought was good, and try not to focus on the existential thoughts that would shake me to my soul (or DNA as it may be). The third option was to find something that was satisfying at an existential level, but had enough substantial evidence and reasoning behind it as to not be stupid.

\begin{center}
If we submit everything to reason, our religion will have no mysterious and supernatural element. If we offend the principles of reason, our religion will be absurd and ridiculous.

-Blaise Pascal
\end{center}

The life of Jesus Christ is where I find that reason. The reason that goes beyond reason. I must note that it does not mean against reason, it goes outside what our reason can see.  Jesus Christ claimed to be the way the truth and the life. That's a bold claim, and I certainly needed some life. His claims about the world, and how we can have a personal relationship with Him were something that, if they are true, would be so unimaginably great that to not giving them a chance would be missing out on what is potentially a free winning lottery ticket. I don't think there's any amount of reasoning I can give that will guarantee that someone will become a believer, and the reason is that some people don't want to accept the possibility that maybe there is something beyond this universe, and maybe there is a Creator that cares about the universe on an individual and personal level. That concept is something that can satisfy me deeply in both my heart and mind. Jesus Christ could point out the hearts of men, and yet he still cared to talk to them and love them. He had the mind to turn the questions back on those trying to pull one over on him, and he had the loving nature that the children would play with him. How great a duplicity! How marvelous is one that satisfies both my mind and my heart. I hope that all of you can find one that gives a deep rich satisfaction that the world is created, and not merely random chance. There's something so much more splendid, so much more meaningful (and that's meaningful in every sense) to have a Creator that is both mathematician and Father. He is both the Creator of awe inspiring physics from the macroscopic to the microscopic, and He is the true source of love. That might be crazy, but if it's true, then it's worth everything. I implore you to consider it, and I mean truly consider it, because it might be the winning lottery ticket.

So, where does all of this place the religious model? It's at the center of everything. It is that from which we can build all other models, and it is what we use to guide us. 

\begin{center}
Amazing grace\\
How sweet the sound\\
That saved a wretch like me \\
I once was lost\\
But now I'm found\\
Was blind, but now I see\\
\medskip
'Twas grace that taught\\
My heart to fear\\
And grace my Fears relieved\\
How precious did\\
That grace appear\\
The hour I first believed\\

Amazing Grace - John Newton
\end{center}

And it certainly has saved a wretch like me. How precious is it to have a savior that reaches out to love us?

You could say that I didn't give any evidence for the Christian model, and that is correct, I didn't give any evidence. That doesn't mean there isn't any. The hardest part of considering the Christian model, and incorporating it into our experience of the world, is not on the evidence. The initial consideration that there is something beyond the material is the hardest consideration. It takes us outside everything we observe on a day to day, but yet gives us a new lens from which we can see everything. The evidence is laid out in the bible, but I think rarely is the evidence the \textit{root} of the problem. I also don't think that it is evidence that will cause people to see Jesus as the Messiah, and that the revelation in itself is an act of God. Jesus questioned Peter and asked who Peter thought Jesus was, and it is very telling on how anyone can have a revelation on who Jesus really is.

\begin{center}
15 ``But what about you?” he asked. ``Who do you say I am?”\\
16 Simon Peter answered, ``You are the Messiah, the Son of the living God.”\\
17 Jesus replied, ``Blessed are you, Simon son of Jonah, for this was not revealed to you by flesh and blood, but by my Father in heaven.

Matthew 16:15-17
\end{center}

So no, I will not be giving an in depth study of the historical findings that are relevant to the historical recordings of the bible. I won't list the miracles He has done throughout time\footnote{Jesus did many other things as well. If every one of them were written down, I suppose that even the whole world would not have room for the books that would be written. John 21:25}, and I won't give a line of reasoning that points to the need for a creator. If you want those arguments then you have to first be open to the concept of the living God, and from that He can work within you.

\section{The Scientific Model}
If the religious model gave me love and everlasting joy, then does that mean the scientific model points me to nihilism and despair? No. To say science has been important throughout human history is, as previously mentioned, an understatement. Going back to throwing rocks instead of fists, throwing spears instead of rocks, and now having mass farming that uses gene manipulation to improve crop production. Where would we be without science? Well probably still throwing fists, but I digress. 

Science is about describing the world in a way that we can predict. Once we can predict it, we can try and account for it, or even try and use it to our advantage. If I figure out how to predict when gasoline will combust, then I can use that to build a car. If I can predict how metals will heat and shape, then I can build incredible sky scrapers. If I can figure out how particles come together, then I can use them to further understand what our universe is made of. If I didn't think science had anything to say, then I wouldn't be studying it. Science shows us some amazing things, and lets us do amazing things.

Can we live entirely within a scientific model? Yes, but we need to take some statements at face value. Science is built on the presupposition that the universe has consistent laws that maintain the predictable nature of the universe.  We also have to assume that everything is random chance and there is no given purpose for our lives. We are only living out the DNA that we are given, and we each have no choice throughout life, and we merely are cogs in a machine going to nowhere.

The first problem is that we can't merely create something from nothing (and when I say nothing I mean nothing). Rarely does this direct question affect my day to day grind at work, but if there is a bigger answer than it might be of such importance that if affects our entire life. So could we ignore this problem and live purely in a scientific model? Yes, but I don't recommend it.

The next problem we face in a purely scientific model is the need for the universe to be explainable. Because we can only ever show that our scientific model is able to work in every scenario we have already observed, how can we show that it will work in every scenario of the universe? We make the assumption that the universe is entirely explainable and predictable. This again is another assumption that must be questioned. Just because all of our observations point to the universe being predictable does not mean that it is. Could we ignore this problem and live purely in a scientific model? Yes, but again, I don't recommend it.

Does this lead me to say you shouldn't live by the scientific model in any capacity? Absolutely not! I'm trying to understand and experiment on the universe just as much as anyone else, which requires the scientific view of predictable and exploitable models of the universe. So how do we balance these two?

\section{The Christian Scientific Harmony}
Being a Christian scientist is quite literally the epitome of having my cake and eating it too. I get to spend all day running simulations, and I can use these models to predict real-world phenomena! That is the coolest thing in the world. I'm excited to say that I contribute to the forefront of human knowledge. Those of us in science might take for granted that we have the greatest job in the world, and with that comes some ethical obligations. I feel it's important to give impartial arguments to my non-technical friends on current problems, and to state when my arguments become opinions. I need to do both because I have the scientific worldview to uphold and share with the rest of the world that doesn't live and work within it.

With my Christian worldview I can see the importance of the world. I can look at the mountains and see their splendor, and be thankful for a Creator who made them. I can have a reason to keep looking at the universe. I could look and explain every known phenomena and have a `perfect model' and still have a reason to continue looking at the universe. I have a reason to get out of bed in the morning. I have a reason that I can look myself in the mirror and love myself. That's powerful. That's personal, and that's the Christian worldview.

So where does this leave us? Do I pick and choose when to invoke one worldview over another? It seems to me that the Christian worldview gives me the overarching story, and that the scientific worldview gives me the material implications of that story. I can look at the material, and that can magnify the greatness of the story. Like observing the fine-tuned parameters of the universe. There are times where the two seem to contradict, and that is where we must really put our eyes to the test to see how the lens' interfere. Miracles are one example where the two views meet in a blinding rainbow. What is amazing to me is that we need the scientific worldview to understand the amazement that is a miracle. If we don't understand how humans emerge from an embryo then we wouldn't see the miracle that is the virgin birth. Miracles only exist because we have science. This is where we then meet and say that the science gives us the statistical significance of an event, but it doesn't give us the big picture (because there is no bigger picture beyond the statistical probabilities in the material world). We can turn to Christ and see what he has said for the deeper meaning.

Plenty of people live their life only by the scientific model, and they must reject occurrences of miracles as just myth or fantasy. They can even live their entire life as a decent civilized human, contributing to society and peacefully die at the age of 85. Like the random dust they profess to have come from, the random dust they shall return. But I will implore you to consider giving your life a deeper meaning. If the possibility might be out there, isn't it worth taking the risk? If I'm honest, I would guess that someone on their suicide bed doesn't particularly want to hear that they are just dust, a mindless cog, or just dancing to their DNA.

\begin{center}
18 I will not leave you as orphans; I will come to you. \\
19 Before long, the world will not see me anymore, but you will see me. Because I live, you also will live. 

John 14:18-19
\end{center}

Are we all on our suicide bed? No, but we all have the same issues that burden our hearts. So I pray, that if you haven't considered the Christian worldview, that you open your heart and see what it really offers. For those who rejected the worldview I implore you to reconsider it. What happens in your life if you genuinely try to include the lens of the Christian worldview? Maybe you still reject Christ, but don't be quick to dismiss Christians in everything they say, for their DNA is just as valuable as yours. 

If you do accept the Christian worldview, then think about the splendor of the universe, and how lucky we are to have a Creator that satisfies on the spiritual and intellectual level.

\begin{comment}
Resources:
First model of the universe : https://lambda.gsfc.nasa.gov/product/suborbit/POLAR/cmb.physics.wisc.edu/tutorial/briefhist.html#:~:text=An%20astronomer%20named%20Eudoxus%20created,the%20Earth%20at%20its%20center.

Sample Sprague Dawley Rats:
https://www.taconic.com/rat-model/sprague-dawley

Older known mathematical theorems (babylonian pythagorean triples)
https://en.wikipedia.org/wiki/Plimpton_322

Galileo and gravity:
https://en.wikipedia.org/wiki/History_of_gravitational_theory#:~:text=1st%20century%20BC)%20understood%20that,based%20on%20their%20specific%20gravity.&text=In%20the%20early%2017th%20century,the%20basic%20principle%20of%20relativity.

Quarks:
https://www.fnal.gov/pub/science/inquiring/questions/quarks.html#:~:text=Nearly%20all%20physicists%20believe%20that%20quarks%20can%20never%20be%20isolated.&text=Eventually%20the%20elastic%20band%20breaks,be%20isolated%20is%20called%20confinement.
\end{comment}
\chapter{The Heart}
We have looked at some ways that I think the Christian worldview has merit in conjunction with the scientific worldview. But one thing I want to emphasize here is the personal nature of our own worldviews. The way I see the world is different from everyone else, and so we each have something new to contribute. Even thought I share similar views with other Christians, it doesn't mean I see the world the exact same.

It's important to take the time to see how art can be an excellent gateway to understanding ourselves. There's a strange fact that we can `learn' things about ourselves. We don't fully know ourselves, and it seems art is an interesting way to access that portion of ourselves. Art is not merely a beautiful painting, or a poem (like the one to follow), but it's an expression and attempt to understand ourselves. Its interesting that when Christ chose to speak truths to us, he used parables. The longest book of the bible (Psalms) is also written as a collection of poetry. 

I've tried to make this book with a bit more of a personal tone on looking at my own worldview in hopes that it might inspire you to take the time to understand your own worldview instead of merely just parrot the ideas fed to us. We are all created individually, and are part of a larger society, so contribute your ideas. So let's look a bit deeper into my own mind, and maybe you can transform them into your own.

\section{My Heart - A Poem}
\begin{flushleft}
Who am i?\\
The question of self-actualization\\
Questioning my self misappropriation\\
Leading to conclusion\\
or possibly self delusion
\bigskip

Am I characterized by actions\\
or should we focus on my character
\bigskip

Now in the 21st century I'm left to question whether my name is worth a mention. 
Will my name be left by curb destined for a landfill, or remembered and told to people I'll never know? \\
Or maybe they will remember me for what i know and nothing more?
\bigskip

All this leads to is existential dread\\
The dread that's only ever misread\\
Because if my analysis is left without bias\\
i can't seem to figure out if the data is worth analysis to begin with
\bigskip

Paul might not have been suicidal but he certainly talked a lot about death\\
If to live is Christ and to die is gain\\
Maybe death is worth leaving this existential pain\\
So give me one ticket on that ol' black train\\
And i'll be on my way
\bigskip

But I know to stay is better\\
So let me write you this letter\\
And maybe I'll go from stop collector to go-getter
\bigskip

Maybe I'll stop sitting on my hands\\
And Maybe I'll start playing in the band\\
Maybe I'll stop running into walls\\
And Maybe I'll start chanting down the halls\\
Maybe I'll stop breaking others down\\
And Maybe I'll start turning them around\\
Maybe I'll stop ruining my life\\
And Maybe I'll start living life for Christ\\
\bigskip

You may not live in existential dread\\
Maybe it's something I only think about in bed\\
But when a new day comes, and I can't seem to stand, I can't help but wonder
\bigskip

Maybe I could do more\\
Maybe...\\
Maybe...\\
Maybe
\bigskip

I need something to make it worth looking in the mirror\\
Something where I'm not holding back the tears\\
You might call it teen angst, or just feeling down\\
But I have to wonder if this is where I drown\\
\bigskip

There's a feeling deep inside my heart\\
A wanting for something expressed with art\\
A feeling deep inside my soul\\
A wanting to fill the empty bowl
\bigskip

It's not a simple equation of my life\\
It's not a mathematical principle to find a wife\\
It's not a physical constant to guide my way\\
It's not science that will make me stay
\bigskip

I will not live ``of blood and soil"\\
I will not live for meaningless toil\\
I will not make light the feelings in my heart\\
I will not merely be  part
\bigskip

A part as a cog in the machine. With no real abilities or means of change, but simply dancing to my DNA. My thinking is a scripted set of throws of the dice meant to masquerade my ability of choice. Neither me nor my friend Will is free to decide their fate.\\
No I would not like this version of reality. I would prefer one that our system is already presupposed on. I want to think I am more than electrical synapses. If I am told that my entirety of life exists within the electrochemical world, then I wonder why I am compared to monkeys and not a computer. Do we really live in a world where we will be left with nothing to signify the difference between robot and human? Maybe that is the correct view, but hopefully deep down there is something that makes us ``in the image of God." Something different than the transistors in a Von Neumann machine. Something beyond the material.
\bigskip

If there is something more\\
If it really is worth the chore\\
Would we take it?
\bigskip

What if Steven Hawking is wrong?\\
What do we say when we feel like we don't belong?\\
What happens when we pull back this materialistic facade?\\
Maybe then... and only then... we might truly know "the mind of God".
\bigskip

It's my biggest wager\\
Upon which I stake my life\\
Not to live for this planet\\
But to be a part of the wife
\bigskip

So when life has beaten you down\\
When nothing has gone your way\\
We can blame chance for ruining your day
\bigskip

But when the last day comes\\
And I draw my final breath\\
I see my life go by\\
Smelling the stench of death
\bigskip

I will be left wondering why\\
Maybe it was all the roll of a die\\
But maybe it's something more\\
Something that might have been worth the chore
\bigskip

Maybe I can stop merely dancing to my DNA\\
Or at least I can pick the tune\\
Maybe I can stop the gears from grinding\\
Or at least make them a little more smooth
\bigskip

Once again I must implore\\
Maybe there is something more\\
Something that speaks\\
To my deepest core
\end{flushleft}

\section{What does it mean?}
One thing about art in general is that the interpretation is left to the reader. This is unlike academic works that go into painful detail about the significance and proper interpretation (of which people may disagree on). Stephen Hawking left no margin of interpretation of his point of view, and what the science has explained. However, art allows the reader to see what they want to see. Similar to the Rorschach test (see \ref{fig:rorschach}), our deepest self is able to manifest in our interpretation of the work. In this way something extrinsic informs us about the intrinsic. That is something that has drawn me toward all kinds of music, and particular artists with beautiful or truthful lyrics. 

\begin{figure}
\includegraphics[width=\linewidth]{./figs/rorschach.jpg}
\caption{First of ten inkblots of the Rorschach Test}
\label{fig:rorschach}
\end{figure}

Often we describe these lyrics that speak to us as resonating with us. When I tune my guitar, and play one string, it actually causes all the other strings to vibrate a bit as well. The interesting part about this is that if two strings are tuned to the same note, and you pluck one, it will cause the other string to vibrate much more than the rest of the strings on the guitar. Because the notes have similar properties (being in some harmonic relationship) they cause each other to resonate. This is a perfect description of how lyrics or any art can resonate within us. I can't necessarily tell exactly how I feel inside, and it's something I can struggle with. But I can find art that resonates with how I feel. If I can help see what the art means, then maybe I can learn about why that piece resonated with me. This is also something much deeper than physical copper strings vibrating, its my metaphysical heart strings that are vibrating. The part of my being that is beyond the neurons. A deeper sense of being itself.

Looking back at the poem I wrote, the question is not what the art means to me, but what does it mean to you? A lot of this contains references to other pieces of media that have formed me (or at the very least have given me some sort of idea or intelligence). I hope that there is at least some portion that speaks out and resonates with you, because then you can learn more about your deeper self.

Did you resonate with verses about feeling worthless? Wondering where you are in life, and where you should be going? Did it stick out to you more than the other portions of the poem? Maybe that is worth thinking about. I mean really sit down and think about yourself, and that's tough to do. The easy way through life is to merely looking single mindedly at the next spot on the hopscotch court. But if you don't think about where the path is going, then how do you know if you are going in the right direction? Or maybe these verses didn't resonate with you, but you can see the value that someone else might attribute to them (and I know I certainly attribute value to them).

Maybe the call to action verses is what resonated with you. Is it because you want to champion for something greater than yourself? It's possible you might be more like me, and spending more time reading books trying to gain wisdom than looking inside yourself for where to go next. Too often we can think, ``if I just read one more book maybe I'll be a little bit smarter and will start to understand myself." But I think we need to take the time to sit down and talk to ourselves. We talk about distractions being an impedance between us and meeting with other people, but the greatest distractions distance us from ourselves. 

The verses about God likely lead to polarizing conclusions. Some people immediately read those verses and agree with Laplace, ``I have no need for that hypothesis," but others feel a deep yearning for something more. To discredit their deeper feelings is a discredit of their being. Regardless, we must each see how we interpret the art. I must understand that some people find no missing piece when trying to put the puzzle of the universe together. They must understand that the black hole that resides in my heart makes me search for something beyond. It's not that I am trying to show why my interpretation is better than someone else's, but it's about understanding the human condition that we all must face. We all have questions deep within us, and I don't think it's easy for anyone to perfectly understand themselves.

So does it matter how I interpret this poem? It means everything to me, because it reveals who I am. Taking the time to step back from the piece, and look into myself is where the value is. The orchestra can play a beautiful note, and it only wells up the emotions that are already inside of me. When the conductor holds out the last note, are not my emotions welling up with the crescendo? When the lyrics of the song talk about my deepest failures, is it meaningful because the meaning was already in me? What does this poem me to me? It depends how deeply it resonates with me. The question that is more important is what does this poem mean to you? And what does this poem mean to those closest around you? I can't tell you what is in yourself. There is no graph I can make that will tell you exactly who you are. I can lay out your entire genome, and still it won't make you understand yourself at your deepest level. Who are you? These are the questions that art can answer for us.

Art gives us understanding that is different than science. Science certainly gives us understanding of the mechanisms at work within the universe. But what about the mechanisms of the mind and soul. There's an argument to be made that everything can be broken down to a scientific explanation eventually, and that broken down a lot of poor religious beliefs throughout the millenia. But the concept of being more than just DNA is something that a lot of people resonate with. We need to understand what it means to look into our mind and soul, and to understand our full being. Is there more beyond us? 

\begin{flushleft}
Once again I must implore\\
Maybe there is something more\\
Something that speaks\\
To my deepest core
\end{flushleft}

\part{The Difference Between Christians and Christ}
\chapter{Christians for the Souls}
\begin{center}
``I like your Christ, I do not like your Christians. Your Christians are so unlike your Christ."

- Mahatma Ghandhi
\end{center}
\begin{figure}[h!]
\begin{center}
\includegraphics[width=0.75\textwidth]{./figs/goodsamaritan.jpg}
\caption{Luke 10:25-37 The Parable of the Good Samaritan painted by Rembrandt Harmenszoon van Rijn.}
\end{center}
\label{fig:goodsamaritan}
\end{figure}
I've spent the last few chapters focusing on why we as academics need to keep an open mind on why Christianity has value, but this chapter I want to shift to really focus on the Christians and their own lives.

When we discuss Christian living, I find myself wanting to put out my own ideas, but I'm going to try and refrain from that. This section will be very dense with scripture. I'm hoping that it is not my truth that speaks, but Christ's truth.

I'm concerned for how some people take Christianity can abuse 1 Peter 3:15 (and ignore 1 Peter 3:16)

\begin{center}
15 But in your hearts revere Christ as Lord. Always be prepared to give an answer to everyone who asks you to give the reason for the hope that you have. But do this with gentleness and respect, \\
16 keeping a clear conscience, so that those who speak maliciously against your good behavior in Christ may be ashamed of their slander.

1 Peter 3:15-16
\end{center}

The common abuse of 1 Peter 3:15 is that we as Christians should have a set of arguments at any point to shoot down any arguments against Christianity and we should do it so that the other persons arguments become stupid. In some sense, when things fully devolve into an argument it is no longer about what is correct but about proving the other person to be wrong, and not actually wrong, but in the sense that they are shamed by the arguments they hold. What part of that is in any way gentle or respectful?  None of it at that point is either.  As soon as the `discussion' becomes a proof by \textit{reductio ad absurdum} of the opponent, we have left gentleness and respect out of the conversation. 

In a world where we are increasing toward arguments instead of discussions, read the passage again, it says be prepared to answer everyone who asks. How often are we giving answers to questions people weren't asking? 

That leaves us with verse 16. Keep a clear conscience, and keep it to such an extent that those who slander will be ashamed. Ashamed! Think about all the times we can fly off the handle. Think about all the times we can bend the truth just a little bit. Think about how often we push our agenda. Do we stand up to the test? Do we really leave our slanderers ashamed.

\begin{center}
3 ``Why do you look at the speck of sawdust in your brother’s eye and pay no attention to the plank in your own eye? 4 How can you say to your brother, ‘Let me take the speck out of your eye,’ when all the time there is a plank in your own eye? 5 You hypocrite, first take the plank out of your own eye, and then you will see clearly to remove the speck from your brother’s eye.

Matthew 7:3-5
\end{center}

Do we as Christians look deeply into ourselves and find our flaws and try and work on them, and I mean really work on them. Spend time being introspective, and analyzing our pitfalls, and trying to live as Christ-like as possible. I don't want Ghandhi to be right about us, but it seems too often that he is. Too often people can poke holes in Christians, even if they can't poke holes in Christ. We need to be showing the love of Christ in such a way that people have to ask why. They have to ask, ``Why are Christians loving others the way they are?" It is then that they can ask for the reason for the hope that is in us. How sad is it when we too often put the cart before the horse.

\section{Practical Christians}
How do we get to the point where slanderers are put to shame? Through constant dedication and love for everyone (for all people are made in the image of God). 

\begin{center}
43 ``You have heard that it was said, ‘Love your neighbor and hate your enemy.’ 44 But I tell you, love your enemies and pray for those who persecute you, 45 that you may be children of your Father in heaven. He causes his sun to rise on the evil and the good, and sends rain on the righteous and the unrighteous. 46 If you love those who love you, what reward will you get? Are not even the tax collectors doing that? 47 And if you greet only your own people, what are you doing more than others? Do not even pagans do that? 48 Be perfect, therefore, as your heavenly Father is perfect.

Matthew 5:43-48
\end{center}

Do you go out of your way to love your enemies? The coworker that slighted you, that person that cut you off in traffic, or the person on the other side of the political spectrum. Do we spend the time to be relational with the people around us? In the US we think a lot about how we need to help those in third world countries, and certainly they need help, but what about the people immediately around us?  Have you really tried to connect with people around you, and I mean \textit{really} try to connect? Everyone has needs, and we are called to meet not just their physical needs, but also their spiritual needs.

We have to then look at how Christ handled his relationships with us.

\begin{center}
5 In your relationships with one another, have the same mindset as Christ Jesus:
6 Who, being in very nature God,
    did not consider equality with God something to be used to his own advantage;
7 rather, he made himself nothing
    by taking the very nature of a servant,
    being made in human likeness.
8 And being found in appearance as a man,
    he humbled himself
    by becoming obedient to death—
        even death on a cross!
9 Therefore God exalted him to the highest place
    and gave him the name that is above every name,
10 that at the name of Jesus every knee should bow,
    in heaven and on earth and under the earth,
11 and every tongue acknowledge that Jesus Christ is Lord,
    to the glory of God the Father.
    
Philippians 2:5-11
\end{center}


How far do you humble yourself before others? Do you go to the full length that Christ did? I know I can fall far short of His example. It's a level of dedication that goes beyond our human abilities, and we need to rely on Him for the strength do love. It is easy to shoot others down to raise ourselves up, but it's hard to lower ourselves to life others up. It is even harder to humble yourself before the person you don't agree with, and to genuinely listen to them.

But if all of this is really true, if we really believe, is it worth standing up to the plate?

\section{The Lukewarm}
I'm concerned that sometimes we don't go all in for Christ. If we know what separates people from eternal life, why are we not constantly going out and talking to people? There are a lot of terrifying passages in the bible, regardless of whether they are metaphoric or literal, they should be terrifying. 

\begin{center}
15 I know your deeds, that you are neither cold nor hot. I wish you were either one or the other! 16 So, because you are lukewarm—neither hot nor cold—I am about to spit you out of my mouth. 17 You say, ‘I am rich; I have acquired wealth and do not need a thing.’ But you do not realize that you are wretched, pitiful, poor, blind and naked. 

Revelation 3:15-17
\end{center}

The Creator of the universe wishes that we were hot or cold instead of lukewarm. The Creator of the universe would wish we were cold instead of lukewarm. When you get comfortable with your life, the Creator of the universe says we are wretched, pitiful, poor, blind, and naked. Think about that, and I mean really stop right now and think about that. Doesn't that shake you to your core.

I still don't think Christians get it, and so here is another passage that should open your eyes.

\begin{center}
24 ``Then the man who had received one bag of gold came. ‘Master,’ he said, ‘I knew that you are a hard man, harvesting where you have not sown and gathering where you have not scattered seed. 25 So I was afraid and went out and hid your gold in the ground. See, here is what belongs to you.’\\
26 ``His master replied, ‘You wicked, lazy servant! So you knew that I harvest where I have not sown and gather where I have not scattered seed? 27 Well then, you should have put my money on deposit with the bankers, so that when I returned I would have received it back with interest.\\
28 ``‘So take the bag of gold from him and give it to the one who has ten bags. 29 For whoever has will be given more, and they will have an abundance. Whoever does not have, even what they have will be taken from them. 30 And throw that worthless servant outside, into the darkness, where there will be weeping and gnashing of teeth.’

Matthew 25:24-30
\end{center}

What servant are we going to be? Are we going to make use of the life on this planet, or are we going to waste it away? Do we take this seriously? Weeping and gnashing of teeth! Just picture what that is like. Don't think of it as ``oh yeah its kinda hot, and its unpleasant." If you aren't left trembling thinking about it, then you haven't really thought about it.

Too often do I find myself not acting, and just thinking, on a subject to decide what the right method is. We can spend hours sitting and thinking about exactly how to interpret Genesis and combine it with cosmological findings. We can spend time working on our logical arguments, so that no matter what atheistic opponent comes against us they will never beat us. But what about going out and living the Word? Maybe we shouldn't focus on making sure every single line of reasoning is perfectly crafted, and we can rely on God for that part while we go and make disciples.

\section{How should we interact?}
So where does this leave us as Christians? How do we stack up? How do we spend our days living for Christ?

\begin{center}
17 So I tell you this, and insist on it in the Lord, that you must no longer live as the Gentiles do, in the futility of their thinking. 18 They are darkened in their understanding and separated from the life of God because of the ignorance that is in them due to the hardening of their hearts. 19 Having lost all sensitivity, they have given themselves over to sensuality so as to indulge in every kind of impurity, and they are full of greed.
20 That, however, is not the way of life you learned 21 when you heard about Christ and were taught in him in accordance with the truth that is in Jesus. 22 You were taught, with regard to your former way of life, to put off your old self, which is being corrupted by its deceitful desires; 23 to be made new in the attitude of your minds; 24 and to put on the new self, created to be like God in true righteousness and holiness.\\
25 Therefore each of you must put off falsehood and speak truthfully to your neighbor, for we are all members of one body. 26 ``In your anger do not sin”: Do not let the sun go down while you are still angry, 27 and do not give the devil a foothold. 28 Anyone who has been stealing must steal no longer, but must work, doing something useful with their own hands, that they may have something to share with those in need.
29 Do not let any unwholesome talk come out of your mouths, but only what is helpful for building others up according to their needs, that it may benefit those who listen. 30 And do not grieve the Holy Spirit of God, with whom you were sealed for the day of redemption. 31 Get rid of all bitterness, rage and anger, brawling and slander, along with every form of malice. 32 Be kind and compassionate to one another, forgiving each other, just as in Christ God forgave you.

Ephesians 4:17-32
\end{center}

But how can we share the gospel? It's by living it out, and loving others. Our testimony should be our strongest reason for others to believe. We should be reaching out and asking people how they are doing, and I mean ask and care about getting a real answer. Everyone is having some struggles in their life. If they aren't personally struggling, then someone in their immediate circle is. It might be possible that everyone in your circle is genuinely doing well, but I doubt it. If you think everyone around you is doing well, then either you have the greatest life (and so does everyone around you) or you haven't really been talking to them. 

How should we engage with those opposing our message? We should openly and honestly explain our beliefs, and share how we see the universe, but shouldn't be forcing our ideas down their throats (as they shouldn't be forcing their ideas onto us). Not even the Father forces us to love Him, and so we shouldn't be forcing our ideas. Leave the judgments to the One who is worthy of judging. We also need to love them, and shouldn't dismiss everything they say. We should be praying for them, and if they are willing we should be talking to them.

\medskip
I don't want Ghandi to be right, but we need to be prepared to give a reason for him to be wrong.


\chapter{The Parable of the PhD}

\begin{figure}[h!]
\includegraphics[width=\linewidth]{./figs/richyoungman.jpg}
\caption{Painting of Christ with the Rich Young man. Described in Matthew 19:16-30 and painted by Heinrich Hofmann}
\label{fig:parable}
\end{figure}

An intelligent man came up to Jesus and said, ``Wise teacher, I have just received my doctorate. I have tried to understand you with all of my mind. What good thing must I know to get eternal life?" \\ ``Why do you call me wise?" Jesus replied. ``There is only One who is wise. If you want to enter life, read the scriptures." \\ ``Which ones?" he inquired. \\ Jesus replied, ``The Torah, the former prophets, the minor prophets, the writings, the Gospels, the letters, and the epistles." \\ ``I have read all of these," the man replied, ``and I have even read the great theologians! Augustine, Lewis, Aquinas, Luther. And I have even begun writings of my own. What do I still lack?" \\ Jesus answered, ``If you want to be perfect, go, leave your proofs and writings. Then come, follow me." \\ When the man heard this, he went back sad, because he had great knowledge. \\ Then Jesus said to his disciples, ``Truly I tell you, it is hard for someone who is academic to enter the kingdom of heaven. Again I tell you , it is easier for a camel to go through the eye of a needle than for someone with a PhD to enter the kingdom of God."

\section{Why is it so hard?}
The average academic has relied on exactly one thing their entire life to make it, their intelligence. The key problem here is that it is \textit{their} intelligence that they rely on. We need to look at what scripture has already said about what we really need to rely on.

\begin{center}
1 My son, do not forget my teaching,
    but keep my commands in your heart,\\
2 for they will prolong your life many years
    and bring you peace and prosperity.\\
3 Let love and faithfulness never leave you;
    bind them around your neck,
    write them on the tablet of your heart.\\
4 Then you will win favor and a good name
    in the sight of God and man.\\
5 Trust in the Lord with all your heart
    and lean not on your own understanding;\\
6 in all your ways submit to him,
    and he will make your paths straight.\\
7 Do not be wise in your own eyes;
    fear the Lord and shun evil.\\
8 This will bring health to your body
    and nourishment to your bones.\\
Proverbs 3: 1-8
\end{center}

The challenge here is that academics rely so hard on their own understanding it becomes increasingly impossible to imagine anything else. Reading scripture feels like a need to beat every sentence to death trying to understand every metaphysical, psychological, and practical implication it has. This simply is not how Christians are meant to live their lives. We are not called to use knowledge as our way to salvation. 

My life as an academic is purely continued by my ability to think. My life as a Christian is purely continued by God's divine grace. There isn't anything for me to understand about it. There isn't a proof for me to build to come to a clear and obvious conclusion. If I claimed to have the exact answer to every single nuanced question about Christ and life, then you should close this book immediately and burn it. The claim to know the mind of God is the exact opposite of what Christ has called us to. 

\section{Faith like a Child}
This is the toughest balance for the academic mind. I don't think we are meant to blindly follow. We are told numerous times of the miracles Christ has done. The miracles themselves are written so that we might believe.

\begin{center}
But these are written that you may believe that Jesus is the Messiah, the Son of God, and that by believing you may have life in his name.

John 20:31
\end{center}

Clearly there is some amount of thought in our relationship with Christ, right? And what about the verse from 1 Peter from which we build from to reason our faith?

\begin{center}
15 But in your hearts revere Christ as Lord. Always be prepared to give an answer to everyone who asks you to give the reason for the hope that you have. But do this with gentleness and respect,

1 Peter 3:15
\end{center}

So how can we balance these ideas with,

\begin{center}
1 At that time the disciples came to Jesus and asked, ``Who, then, is the greatest in the kingdom of heaven?”\\
2 He called a little child to him, and placed the child among them. 3 And he said: ``Truly I tell you, unless you change and become like little children, you will never enter the kingdom of heaven. 4 Therefore, whoever takes the lowly position of this child is the greatest in the kingdom of heaven. 5 And whoever welcomes one such child in my name welcomes me.

Matthew 18:1-5
\end{center}

So which is it? Are we suppose to understand with our minds, or with our hearts. I think this is not an easy answer, because I think clearly we are suppose to believe with both. We are called to, 

\begin{center}
Love the Lord your God with all your heart and with all your soul and with all your mind and with all your strength.

Mark 12:30
\end{center}

We need to love God with both our heart and mind, but how do we do this practically. I think it might be better to start with the heart. My answer is not because I see a clear biblical necessity to start with the heart, but because conversely it is easy to stop at the mind. I genuinely believe that you either start choosing to believe in your heart, and you then build from whichever choice you have made (as I have with my choice). I started with an open and willing heart, and with that I can then be like Peter.

\begin{center}
13 When Jesus came to the region of Caesarea Philippi, he asked his disciples, ``Who do people say the Son of Man is?”\\
14 They replied, ``Some say John the Baptist; others say Elijah; and still others, Jeremiah or one of the prophets.”\\
15 ``But what about you?” he asked. ``Who do you say I am?”\\
16 Simon Peter answered, ``You are the Messiah, the Son of the living God.\\
17 Jesus replied, ``Blessed are you, Simon son of Jonah, for this was not revealed to you by flesh and blood, but by my Father in heaven. 18 And I tell you that you are Peter, and on this rock I will build my church, and the gates of Hades will not overcome it.

Matthew 16:13-18
\end{center}

For better or for worse, God gets to be God. Often times we act as if the salvation is dependent on us, but we need to know it is all dependent on Him. We are merely the channel through which it occurs. We can be a willing part of the process, the clay from which the sculptor can mold, or we can reject that and try and mold ourselves. This does not, and cannot, mean that we do not have the ability to use our minds in the process. We are clearly called, as written in the passage from Mark 12, to love the Lord with our minds as well. Yes, go ahead and try and understand what God is saying to you. Seek wisdom on what God has already planned for you. Go ahead, don't leave your mind at the door, but make sure you are using Christ as the key to unlock it, and your heart to open it.

\section{Get out of your head!}
To often we feel the need to beat every question we can come up with to death before we come to a conclusion. But we need to question, are we asking every question to try and understand more or because we don't want to believe what we know the conclusions might lead to. Regularly Christ always turns the question back on those who are asking him. Looking at the parable of the PhD, we see a man who is full of knowledge. He knows Christ is good in his mind, but hasn't applied that in his heart. He started with his mind, and then tried to use that to save his heart.

What do you cling to first, the Gospels, or the next deep theological book from your favorite scholar? If I'm honest, I tend to want to read from my favorite writers. What insanity is that? I could be reading what I claim is the Word of God, and instead I want something that happens to match my preferred flavor of academic stimulation. We have books that are written to give a deep dive on a single question, on an a portion of doctrine that isn't an absolute necessity to understand. And we tend to want to read these over what the Creator of the universe has spoken to us.

Why can we not leave the simple stuff simple? Maybe I sound like I'm throwing the idea of reason out the window, but hopefully that isn't your takeaway. We need to make sure that we aren't a pure set of facts and proofs to create our beliefs, and instead we open our hearts and let God take the lead (He's pretty good at it).

\section{The Simple Message}
One thing that has really been tough for me personally, is making sure I focus on the basics of exactly what is already written in the Bible, as opposed to reading from some other author. There are plenty of excellent theologians, apologists, and other Christian writers that have covered countless pages in text. But do we use these as a substitute to what we claim to be the Word of God?

Do we get to a point where we start to make our relationship with God merely a search for a perfect theology? This doesn't mean theology should be thrown out the window, but we shouldn't let theology be our God. It's easy to want to dive into deep discussions with fellow Christians, but what about just being in awe of the sacrifice Christ gave. We need to be sure that while we certainly can have deep discussions on the interpretation of predestination, we don't let that stop us from living out the simple things Christ has clearly called us to. 

Sometimes our own understanding can be a crutch for a lack of faith. Let it not be so. If we truly believe the Creator of the universe is on our side, then sometimes we need to believe that he is the one watching over our battles. Theology can help us understand and piece together more of God's plan, but we can't let theology become our God. We don't worship ideas. We worship Christ. 

\begin{center}
8 ``For my thoughts are not your thoughts,
    neither are your ways my ways,”
declares the Lord.\\
9 ``As the heavens are higher than the earth,
    so are my ways higher than your ways
    and my thoughts than your thoughts.

Isaiah 55:8-9
\end{center}

After all, God is God. If we were really suppose to sit and think until we knew all of the answers, then we would never actually go out and live what we are called to. The disciples didn't sit under a tree and try and go back and piece together every single thing Christ had said. They went out and lived what Christ called them to do, and relied on the Holy Spirit to guide them. We need to do the same.

\chapter{Now What?}
I think there is a butting of heads between Christians and science. As a Christian scientist this puts me in the middle of two groups that don't know how to talk to each other, but are constantly pointing fingers at the other. I have plenty of Christians who don't know how to talk to scientists, and how they should interact and view them. And I know plenty of scientists that don't know how to talk with Christians. We need to figure out how to have a decent dialogue and work together (which doesn't mean we will agree). There are points when science and Christianity overlap and we need to know how to handle those discussions. Too often I see Christians trying to force conversions of scientists, and scientists trying to make Christians look like lunatics, and neither of them help (even when they might be trying to do what is right). I'm hoping that as someone who is in both camps, I can try to bridge the gap a little bit. One book will never solve all of our problems (even the bible is 73 books), but hopefully it can help.

\section{For the Scientist}
Some of the previous chapters may sound crazy, and you might believe that it is completely crazy, and I doubt I will ever change that. My hope is that we as scientists can understand how to interact with Christians. Few people will have the knowledge that we possess on our particular subject. Being a scientist is a high prestige position that we hold, and so we need to talk about the ramifications of our actions.

I hope that we as scientists can see where Christians come from and understand how we can interact with them in a way that is effective at communicating our messages. We need to watch how far we extrapolate our results because many people don't know when we stretch the truth. We typically make up the intellectual elite, so what we have been blessed to easily understand, others cannot. The average IQ is 100 and makes a smooth bell curve, which means every person with an IQ over 100 there is some person with an IQ below 100. So how do we make sure that when we have important concepts we make them clear? Too often we are left with the picture in our heads of the professor who always gave abysmal lectures, and never gave any interesting material to anyone other than those working directly around him. We have ideas, and we need to ensure that our ideas are made both for our fellow academic, and for our fellow man.

Hopefully we can also take up the ethical standards that our job requires. Understanding the ramifications of what we create is an important consideration that is ever needed as we approach more dangerous developments. Our theological counterparts might have some ideas that are worth considering (or at least listening to) in our understanding of how to keep the world together. Many of us have tools and skills that could bring down significant terror. Whether that is working with extremely dangerous virus's, or working on the next machine learning algorithm for hacking, we need to do it with a respect for the position we are in.

\section{For the Christian}
We claim to be children of God called to share the Good News, but how well do we really do that? I tried to lay out a challenge to live up to the calling we claim. Christ was the incarnation of the living God sent to atone for the sins of the world, and we are suppose to follow Him. 

\begin{center}
22 If I had not come and spoken to them, they would not be guilty of sin; but now they have no excuse for their sin.

John 15:22
\end{center}

I'm tried to lay out some of the bold claims of what God has said, and they should cause us to want to live our lives differently. Living for Christ is the greatest reason for living I can come up with. I have so much joy everyday getting to know I have a creator above me. When my life has meaning things can only go up (even when they go down I have a reason to keep going). Because of this meaning, I am called to ensure that I speak the truth (for I am made in God's image), and so I hope that there is clarity when I am speaking my opinions and the Truth.

We also need to step back and think about the gravity of the beliefs we hold. I believe that the Creator of the Universe wants to have a personal relationship with me. Even more I believe that he has paid the ultimate price as a loving gesture to us in spite of all of our inequities. If we really believe that, then the sheer weight of that should bring us to our knees. Too often we can become callous to the absolutely awe inspiring beauty of the world around us and how lucky we truly are. Let us not grow callous.

\begin{center}
11 But you, man of God, flee from all this, and pursue righteousness, godliness, faith, love, endurance and gentleness. 12 Fight the good fight of the faith. Take hold of the eternal life to which you were called when you made your good confession in the presence of many witnesses. 13 In the sight of God, who gives life to everything, and of Christ Jesus, who while testifying before Pontius Pilate made the good confession, I charge you 14 to keep this command without spot or blame until the appearing of our Lord Jesus Christ, 15 which God will bring about in his own time—God, the blessed and only Ruler, the King of kings and Lord of lords, 16 who alone is immortal and who lives in unapproachable light, whom no one has seen or can see. To him be honor and might forever. Amen.

1 Timothy 6:11-16
\end{center}

Don't just go through the motions. Stop, think, and questions the world, but in the end we must rely on him. I don't think we are meant to know everything, but luckily we don't need to. Maybe we don't need to keep trying to understand every single thing about God. Maybe it's for the best that we let God be God (I think He's doing a pretty good job), and we focus on following Him.

\section{For All of Us}
We are growing increasingly polarized for a vast number of reasons. Partly because it is so easy for us to stick only to what we know and surround ourselves with people who match us. I also think its because we are quick to want to put our `correct' ideas onto other `uniformed' people. The biggest problem is that we speak without wanting to listen. Even if people say things that are 90\% wrong, then that means 1 in every 10 things they say might be something you don't know. It's a good thing for us to remember we aren't perfect. We don't know everything, and the likelihood that someone you don't agree with knows something you don't is extremely high.

I tried to lay out my life. Not because I was trying to convince you of anything, but because I want others to try and understand my point of view. It's not that I require you to be a Christian for us to converse, but it's that my Christian life means so much to me, and it has so much value to me that I want others to be able to see that. It's not that I'm forcing others to only see things the way I see them, but rather to at least try and see how we can each see the other point of view.

\begin{figure}[hb!]
\includegraphics[width=\linewidth]{./figs/elephant.jpg}
\caption{We all have a different perspective. It doesn't mean we are all necessarily wrong or right, but we are all smarter if we work together}
\label{fig:elephant}
\end{figure}


\appendix
\chapter{Hyatt Walkway Collapse Victims}
\label{appendix:hyatt}
To give a fuller magnitude of how many lives can be affected by a minor mistake. Here is a list that was published in the New York Times.

\begin{verbatim}
ADLER, John J., 74 years old, Fairway, Kan. 
ALCALA, Connie, 50, Topeka, Kan. 
ALLEN, Velma, 49, Kansas City, Mo. 
ANDREW, Carol. 33. 
BARTELS, Bonnie, 36, Lenexa, Kan. 
BARTELS, William 3d, 38, Lenexa, Kan. 
BARTON, Robert, 56, Kansas City, Mo. 
BENTEAU, Robert C., 49, Parkville, Mo. 
BERGES, Calvin W., 57, Kansas City, Mo. 
BERGES, Florence, 62, Kansas City, Mo. 
BERGMAN, John W., 30, Shawnee, Kan. 
BERGMAN, Pearl, 58, Shawnee, Kan. 
BOLTON, James, 63, Chesterfield, Mo.
BOLTON, Julia, 50, Chesterfield, Mo. 
BOTNEN, Henry, 51, Overland Park, Kan. 
BOTTENBERG, Louis, 67, 
BROOKS, Jaqueline, 24, Kansas City, Mo. 
BURGESS, Florence, 62, Kansas City, Mo. 
CARMONA, Delores, 35, Topeka, Kan. 
CAST, Theodore, 72, Holden, Mo. 
COFFEY, Gerland, 42, Leavenworth, Kan.
COFFEY, Pamela, 11, Leavenworth, Kan. 
COTTINGHAM, James, 46, Kansas City, Mo. 
DAUGHERTY, James, 56, Merriam, Kan. 
DAUGHERTY, Barbara, 51, Merriam, Kan. 
DePRIEST, Christina, 22, Carney, Mo. 
DEKROYFF, Richard, 56, Kansas City, Mo. 
DETRICK, Calvin Jr., 66, Kansas City, Mo. 
DIAL, Clifton, 80, Portland, Ore. 
DUNCAN, Lois, 62, Excelsoir Springs, Mo. 
DURHAM, Jeff, 26, Paola, Kan. 
FARRIS, Louis, 42, St. Joseph, Mo. 
FIENE, Carol, 48, Gladstone, Mo. 
GALVIN, Delores, 26, Topeka, Kan. 
GLASER, John, 58, Kansas City, Mo. 
GLOVER, Laurett, 55, Merriam, Kan. 
GLOVER, Ray, 56, Merriam, Kan. 
GOSS, Richard, 42, Kansas City, Mo. 
GRIGSBY, Roger, 38, Kansas City, Mo. 
GRUENING, Jean, 48, Prairie Village, Kan. 
GRUENING, William, 48, Prairie Village, Kan. 
GUBAR, Joseph, 56, Kansas City, Mo. 
HACKETT, Virginia, 66, Kansas City, Mo. 
HANSEN, Paul, 51, Mission, Kan. 
HAZELBECK, Mary, 56, Overland Park, Kan. 
HENSON, Thomas, 49, Independence, Mo. 
HENSON, Remelia, 29, Independence, Mo. 
HERSHMAN, Stephen, 59.
HILL, Doris, 56, Lenexa, Kan. 
HILL, Forest, 56, Lenexa, Kan. 
HOULTBERG, Richard, 53, Overland Park, Kan. 
HUNTSUCKER, Carl, 44, Raytown, Mo. 
JETER, Eugene, 48, Kansas City, Mo. 
JETER, Karen, 37, Kansas City, Mo. 
JOHNSON, Jean, Kansas City, Mo. 
JONAS, Robert, 58, Overland Park, Kan. 
KOLEGA, Elizabeth, 58, St. Joseph, Mo. 
LAMAR, Julia, 33, Overland Park, Kan. 
LANE, Mary, 57, Kansas City, Mo. 
LONGMOORE, Bill, 54, Overland Park, Kan. 
McCLELLAN, Clara, 57. 
McLANE, Betty, 57, Prairie Village, Kan. 
McLANE, William, 57, Prairie Village, Kan. 
MILLER, Betty, 55, St. Joseph, Mo. 
MILLER, David, 51, Overland Park, Kan. 
MITCHELL, Vernon, 52, Independence, Mo. 
MOBERG, Susan. 
MORGAN, Sheryl, 33, St. Joseph, Mo. 
MORRIS, Marjorie, 48, Overland Park, Kan.
NOBLE, Nick, Independence, Mo. 
O'CONNOR, Louise, 62, Mission, Kan. 
O'CONNOR, Neal, 63, Mission, Kan. 
OMER, Leona, Grady, Colo. 
PAOLOZZI, James, 39, 6206 Morningside Drive, Kansas City, Mo. 
RAU, Jerold, 42, 6204 Morningside Drive, Kansas City, Mo. 
RINEHART, Paul, 46, Overland Park, Kan. 
RODMAN, John, 78, Kansas City, Mo. 
SCANLON, Ruby, 54, Overland Park, Kan. 
SCURLOCK, Linda, Topeka, Kan. 
SHOLTS, Floyd, 69, Kansas City, Mo. 
SHOLTS, Violet, 55, Kansas City, Mo. 
SIGLER, Ruth, 57, Kansas City, Mo. 
SIGLER, William, 61, Kansas City, Mo. 
STARK, Helen, 26, 8652 Sleepy Hollow, Kansas City, Mo. 
STEIN, Edmund, 68, Overland Park, Kan. 
STEIN, Viola, 64, Overland Park, Kan.
STOVER, David, 49, 1835 W. Third, Dubuque, Iowa. 
SULLIVAN, Kathryn, Blue Springs, Mo. 
TAYLOR, Lucille, 69, Kansas City, Mo. 
TERRY, Ann, 54, Kansas City, Kan. 
TORREY, Mary, 48, Roeland Park, Kan. 
TORREY, Robert, 53, Roeland Park, Kan.
TVEDTEN, John, Kansas City, Mo. 
VANDERHAYDEN, Lynn, 22, Shawnee, Kan. 
WALSH, Karyn, 42, Kansas City, Mo. 
WATSON, Lawrence, 38, Parkville, Mo. 
WATSON, Suzannie, 43, Parkville, Mo. 
WHARTON, Linda, 26, Lake Quivira, Kan. 
WHITNEY, Edward, 56, Raymore, Mo. 
WHITNEY, Joyce, 50, Raymore, Mo. 
WICKER, Ferna, 52, Overland Park, Kan. 
WILBER, Kathleen, 55, Overland Park, Kan. 
WILLIAMS, James. 42, Oak Grove, Mo. 
WINETT, Paul, 38, Herington, Kan. 
ZATEZALO, Rudolph E., 60, Kansas City, Mo.
\end{verbatim}


\bibliographystyle{plain}
\bibliography{./bibliography.bib}

\end{document}
